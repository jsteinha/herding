\documentclass[final]{beamer}
\mode<presentation>
{
 %\usetheme{Berlin}
 % \usetheme{Aachen}
%  \usetheme{Oldi6}
%  \usetheme{I6td}
 \usetheme{I6dv}
%  \usetheme{I6pd}
%  \usetheme{I6pd2}
}

\usepackage{bm}

\usepackage{times}
\usepackage{amsmath,amssymb,amsthm}

\usepackage[english]{babel}
\usepackage[latin1]{inputenc}

\usepackage{graphicx}
\usepackage{tabularx}
\usepackage{natbib}


\usepackage{hyperref,url}
\usepackage{amsmath,amssymb,amsthm}
\usepackage{tikz}
%\usepackage{float,subcaption,graphicx}
\usepackage{stmaryrd,wasysym,clrscode}
\usepackage{etex,etoolbox}
\usepackage{ifthen}
\usetikzlibrary{patterns,positioning}

%%% Added by Jonathan %%%
\def\[#1\]{\begin{align}#1\end{align}}
\def\(#1\){\begin{align*}#1\end{align*}}
\newcommand{\ip}[2]{\left\langle #1, #2 \right\rangle}
\definecolor{NAColor}{rgb}{.75,0,.75}
\newcommand{\NA}[1]{\textcolor{NAColor}{($\star$) #1}}
\newcommand{\dee}{\mathrm{d}}
\newcommand{\algname}[1]{\textsc{\lowercase{#1}}}
\def\argmax{\operatornamewithlimits{arg\,max}}
\def\argmin{\operatornamewithlimits{arg\,min}}
\newcommand{\defined}{\ensuremath{\triangleq}}
\newcommand{\bprf}{\begin{proof}}
\newcommand{\eprf}{\end{proof}}
\newcommand{\blem}{\begin{lemma}}
\newcommand{\elem}{\end{lemma}}
\newcommand{\eps}{\epsilon}
%%% End added by Jonathan %%%

\newcommand{\lc}[1]{#1_{\mathrm{loc}}}
\newcommand{\eq}[1]{\stackrel{\mathrm{#1}}{=}}
\DeclareMathOperator{\Var}{Var}
\DeclareMathOperator{\sign}{sign}
\DeclareMathOperator{\MMD}{MMD}
\newcommand{\MMDr}{\tilde{\MMD}}
\DeclareMathOperator{\Tr}{Tr}
\newcommand{\inner}[2]{\langle #1, #2 \rangle}
\newcommand{\E}{\mathcal{E}}
\newcommand{\eqdef}{\stackrel{\mathrm{def}}{=}}
\newcommand{\bP}{\mathbb{P}}
\newcommand{\bI}{\mathbb{I}}
\newcommand{\bE}{\mathbb{E}}
\newcommand{\sF}{\mathcal{F}}
\newcommand{\sH}{\mathcal{H}}
\newcommand{\sC}{\mathcal{C}}
\newcommand{\sM}{\mathcal{M}}
\newcommand{\sE}{\mathcal{E}}
\newcommand{\C}{\mathcal{C}}
\newcommand{\sB}{\mathcal{B}}
\newcommand{\bR}{\mathbb{R}}
\newcommand{\bN}{\mathbb{N}}
\newcommand{\bZ}{\mathbb{Z}}
\newcommand{\sI}{\mathcal{I}}
\newcommand{\sP}{\mathcal{P}}
\newcommand{\sX}{\mathcal{X}}
\newcommand{\sS}{\mathcal{S}}
\newcommand{\sJ}{\mathcal{J}}
\newcommand{\sR}{\mathcal{R}}
\newcommand{\sN}{\mathcal{N}}
\newcommand{\meet}{\wedge}
\newcommand{\RE}[2]{\operatorname{RE}\left(#1 \ \| \ #2\right)}
\newcommand{\KL}[2]{\operatorname{KL}\left(#1 \ \| \ #2\right)}
\newcommand{\KLm}[2]{\operatorname{KL}_m\left(#1 \ \| \ #2\right)}
\newcommand{\score}[2]{\operatorname{score}\left(#1 \ \| \ #2\right)}
\newcommand{\phih}{\hat{\phi}}
\newcommand{\psih}{\hat{\psi}}
\DeclareMathOperator{\supp}{supp}
\DeclareMathOperator{\loc}{loc}
\DeclareMathOperator{\lub}{lub}
\newcommand{\atom}[1]{#1^{\circ}}
\newcommand{\stitch}[2]{\overline{#1}^{#2}}
%\DeclareMathOperator{\argmin}{argmin}
%\DeclareMathOperator{\argmax}{argmax}

\newtheorem{theorem}{Theorem}[section]
\newtheorem{lemma}[theorem]{Lemma}
\newtheorem{proposition}[theorem]{Proposition}
\newtheorem{corollary}[theorem]{Corollary}
\newtheorem{assumption}[theorem]{Assumption}
\theoremstyle{definition}
\newtheorem{example}[theorem]{Example}
\newtheorem{definition}[theorem]{Definition}
\newtheorem{remark}[theorem]{Remark}
\newtheorem{property}[theorem]{Property}

\def\ci{\perp\!\!\!\perp}


\usepackage[orientation=landscape,size=a0,scale=1.4]{beamerposter}
%\usepackage[orientation=landscape,size=custom,width=129,height=110,scale=1.4]{beamerposter}

% Display a grid to help align images
%\beamertemplategridbackground[1cm]

\title%[] %(optional, use only with long paper titles)
{A Greedy Framework for First-Order Optimization}

%\subtitle {Include Only If Paper Has a Subtitle}

\author%[Author, Another] % (optional, use only with lots of authors)
{Jonathan Huggins$^{*1}$, Jacob Steinhardt$^{*2}$}

\institute[] % (optional, but mostly needed)
{$^{*}$Both authors contributed equally to this work. $^1$Massachusetts Institute of Technology. $^{2}$Stanford University.}

\date{December 2013}

\begin{document}

\begin{frame}{} 
%\titlepage  
%\vfill
\begin{columns}
\begin{column}{0.48\linewidth}

%%%%%%%%%%%%%%%%%%%%%%%%%%%%%%%%%%%%%%%%%%%%%%%%%%
%%%% COLUMN 1		 						%%%%%%
%%%%%%%%%%%%%%%%%%%%%%%%%%%%%%%%%%%%%%%%%%%%%%%%%%

%%%%%%%%%%%%%%%%%%%%%%%%%%%%%%%%
%%% BEGIN INTRODUCTION BLOCK %%%
%%%%%%%%%%%%%%%%%%%%%%%%%%%%%%%%
\begin{block}{\large Introduction}
Recent work has shown many connections between conditional gradient 
and other first-order optimization methods, such as herding \cite{Bach:2012a} 
and subgradient descent \cite{Bach:2012b}. By considering a type of 
\emph{proximal conditional method}, which we call boosted mirror 
descent (\bmd), we are able to unify all of these algorithms into 
a single framework, which can be interpreted as taking successive 
arg-mins of a sequence of surrogate functions. Using a standard 
online learning analysis based on Bregman divergences, we are 
able to demonstrate an $O(1/T)$ convergence rate for all algorithms 
in this class. 
\end{block}
%%%%%%%%%%%%%%%%%%%%%%%%%%%%%%
%%% END INTRODUCTION BLOCK %%%
%%%%%%%%%%%%%%%%%%%%%%%%%%%%%%

%%%%%%%%%%%%%%%%%%%%%%%%%%%%%
%%% BEGIN ALGORITHM BLOCK %%%
%%%%%%%%%%%%%%%%%%%%%%%%%%%%%
\begin{block}{\large The Boosted Mirror Descent Algorithm}

More concretely, suppose that we are given a function 
$L : U \times \Theta \to \bR$ defined by
\[ L(u, \theta) = h(u) + \ip{u}{\theta} - R(\theta) \]
and wish to find the \emph{saddle point} 
\[ L_* \eqdef \min_{u} \max_{\theta} L(u, \theta). \]
We should think of $u$ as the primal variables and $\theta$ 
as the dual variables; we will assume throughout that 
$h$ and $R$ are both convex. We will also abuse notation 
and define $L(u) \eqdef \max_{\theta} L(u, \theta)$; we 
can equivalently write $L(u)$ as 
\[ L(u) = h(u) + R^*(u), \]
where $R^*$ is the Fenchel conjugate of $R$. Note that 
$L(u)$ is a convex function. Moreover, since 
$R \leftrightarrow R^*$ is a one-to-one mapping, for 
\emph{any} convex function $L$ and \emph{any} decomposition 
of $L$ into convex functions $h$ and $R^*$, we get a 
corresponding two-argument function $L(u, \theta)$. 

\begin{columns}[t]
%%% PRIMAL ALGORITHM SUB-BLOCK %%%
\begin{column}{0.45\linewidth}
\begin{block}{Primal Algorithm}
Given the function $L(u, \theta)$, we define the following 
optimization procedure, which will generate a sequence of 
points $(u_1,\theta_1), (u_2, \theta_2), \ldots$ converging 
to a saddle point of $L$. First, take a sequence of weights
$\alpha_1, \alpha_2, \ldots$, and for notational convenience 
define
\(
\hat{u}_t = \frac{\sum_{s=1}^t \alpha_su_s}{\sum_{s=1}^t \alpha_s} 
\quad \text{and} \quad 
\hat{\theta}_t = \frac{\sum_{s=1}^t \alpha_s\theta_s}{\sum_{s=1}^t \alpha_s}.
\)
Then 
the {\em primal boosted mirror descent} (\primal) algorithm is:
\begin{enumerate}
\item $u_1 \in \arg\min_u h(u)$
\item $\theta_{t} \in \arg\max_{\theta \in \Theta} \langle \theta, u_t \rangle - R(\theta) = \partial R^{*}(u_{t})$
\item $u_{t+1} \in \arg\min_{u} h(u) + \langle \hat{\theta}_t, u \rangle = \partial h^{*}(-\hat\theta_{t})$
\end{enumerate}
As long as $h$ is strongly convex, for the proper choice of $\alpha_{t}$ we 
obtain the bound (see Corollary~\ref{cor:method-1}):
\begin{equation}
\sup_{\theta \in \Theta} L(\hat{u}_T, \theta) \leq L_{*} + O(1/T).
\end{equation}

As an example, suppose that we are given a $\gamma$-strongly convex function 
$f$: that is, $f(x) = \frac{\gamma}{2}\|x\|_2^2 + f_0(x)$, where $f_0$ is convex. 
Then we let $h(x) = \frac{\gamma}{2}\|x\|_2^2$, $R^*(x) = f_0(x)$, and obtain 
the updates:
\begin{enumerate}
\item $u_1 = 0$
\item $\theta_t = \partial f_0(u_t)$
\item $u_{t+1} = -\frac{1}{\gamma}\hat{\theta}_t = -\frac{\sum_{s=1}^t \alpha_s\partial f_0(u_s)}{\gamma \sum_{s=1}^t \alpha_s}$
\end{enumerate}
We therefore obtain a variant on subgradient descent where $u_{t+1}$ 
is a weighted average of the first $t$ subgradients (times a step size 
$1/\gamma$). Note that these are the subgradients of $f_0$, which are related 
to the subgradients of $f$ by $\partial f_0(x) = \partial f(x) - \gamma x$.
\end{block}
\end{column}

%%% DUAL ALGORITHM SUB-BLOCK %%%
\begin{column}{0.45\linewidth}
\begin{block}{Dual Algorithm}
We can also consider the dual form of our mirror descent algorithm 
({\em dual boosted mirror descent}, or \dual):
\begin{enumerate}
\item $\theta_1 \in \arg\min_{\theta} R(\theta)$a
\item $u_t \in \arg\min_{u} h(u) + \langle \theta_t, u \rangle = \partial h^{*}(-\theta_{t})$
\item $\theta_{t+1} \in \arg\max_{\theta \in \Theta} \langle \theta, \hat{u}_t \rangle - R(\theta) = \partial R^{*}(\hat u_{t})$
\end{enumerate}
Convergence now hinges upon strong convexity of $R$ rather than 
$h$, and we obtain the same $1/T$ convergence 
rate (see Corollary~\ref{cor:method-2}):
\[ \sup_{\theta \in \Theta} L(\hat{u}_T, \theta) \leq L_{*} + O(1/T). \]
An important special case is $h(u) = \left\{ \begin{array}{ccl} 0 & : & u \in K \\ \infty & : & u \not\in K \end{array} \right.$, where $K$ is some 
compact set. Also let $R^* = f$, where $f$ is an arbitrary strongly convex 
function. Then we are minimizing $f$ over the compact set $K$, and we obtain 
the updates from conditional gradient:
\begin{enumerate}
\item $\theta_1 = \partial f(0)$
\item $u_t \in \arg\min_{u \in K} \langle \theta_t, u \rangle$
\item $\theta_{t+1} = \partial f(\hat{u}_t)$
\end{enumerate}
Our notation is slightly different from previous presentations in 
that we use linear weights ($\alpha_t$) instead of geometric weights 
(often denoted $\rho_t$, as in \cite{Bach:2012b}). However, under the 
mapping $\alpha_t = \rho_t/\prod_{s=1}^t (1-\rho_s)$, we obtain an 
algorithm equivalent to the usual formulation.
\end{block}
	
\end{column}
\end{columns}
\end{block}
%%%%%%%%%%%%%%%%%%%%%%%%%%%
%%% END ALGORITHM BLOCK %%%
%%%%%%%%%%%%%%%%%%%%%%%%%%%
\end{column}		

%%%%%%%%%%%%%%%%%%%%%%%%%%%%%%%%%%%%%%%%%%%%%%%%%%
%%%% COLUMN 2		 						%%%%%%
%%%%%%%%%%%%%%%%%%%%%%%%%%%%%%%%%%%%%%%%%%%%%%%%%%

\begin{column}{0.48\linewidth}

%%%%%%%%%%%%%%%%%%%%%%%%%%%%%%%
%%% BEGIN CONVERGENCE BLOCK %%%
%%%%%%%%%%%%%%%%%%%%%%%%%%%%%%%
\begin{block}{\large Convergence of Boosted Mirror Descent}
\begin{columns}[t]
%%% PRIMAL CONVERGENCE SUB-BLOCK %%%
\begin{column}{0.45\linewidth}
\begin{block}{Primal Algorithm}
\begin{theorem}
\label{thm:method-1}
Suppose that $h$ is strongly convex with respect to a norm $\|\cdot\|$ 
and let $r = \sup_{\theta} \|\theta\|_{*}$. Then 
\[ \sup_{\theta} L(\hat{u}, \theta) \leq \sup_{\theta} L(u^*, \theta) + \frac{2r^2}{A_T} \sum_{t=1}^T \frac{\alpha_{t+1}^2A_t}{A_{t+1}^2}. \]
\end{theorem}
\begin{corollary} 
\label{cor:method-1}
Under the hypotheses of Theorem \ref{thm:method-1}, for $\alpha_{t} = 1$ we have
\[ \sup_{\theta} L(\hat{u}, \theta) \leq \sup_{\theta} L(u^*, \theta) + \frac{2r^2 (\log (T) + 1)}{T}. \]
and for $\alpha_t = t$ we have
\[ \sup_{\theta} L(\hat{u}, \theta) \leq \sup_{\theta} L(u^*, \theta) + \frac{8r^2}{T}. \]
\end{corollary}
\end{block}
\end{column}

%%% DUAL CONVERGENCE SUB-BLOCK %%%
\begin{column}{0.45\linewidth}
\begin{block}{Dual Convergence}
\begin{theorem}
\label{thm:method-2}
Suppose that $R$ is strongly convex with respect to a norm $\|\cdot\|$ 
and let $r = \sup_{u} \|u\|_{*}$. Then 
\[ \sup_{\theta} L(\hat{u}, \theta) \leq \sup_{\theta} L(u^*, \theta) + \frac{2r^2}{A_T} \sum_{t=1}^T \frac{\alpha_{t+1}^2A_t}{A_{t+1}^2}. \]
\end{theorem}
\begin{corollary}
\label{cor:method-2}
Under the hypotheses of Theorem \ref{thm:method-2}, for $\alpha_t = 1$ we have
\[ \sup_{\theta} L(\hat{u}, \theta) \leq \sup_{\theta} L(u^*, \theta) + \frac{2r^2(\log(T) + 1)}{T} \]
and for $\alpha_t = t$ we have
\[ \sup_{\theta} L(\hat{u}, \theta) \leq \sup_{\theta} L(u^*, \theta) + \frac{8r^2}{T} \]
\end{corollary}
\end{block}
\end{column}
\end{columns}
\end{block}
%%%%%%%%%%%%%%%%%%%%%%%%%%%%%
%%% END CONVERGENCE BLOCK %%%
%%%%%%%%%%%%%%%%%%%%%%%%%%%%%

%%%%%%%%%%%%%%%%%%%%%%%%%%%%%%%%%%%
%%% BEGIN AHK APPLICATION BLOCK %%%
%%%%%%%%%%%%%%%%%%%%%%%%%%%%%%%%%%%
\begin{block}{\large Application: Solving SDPs}


\end{block}
%%%%%%%%%%%%%%%%%%%%%%%%%%%%%%%%%
%%% END AHK APPLICATION BLOCK %%%
%%%%%%%%%%%%%%%%%%%%%%%%%%%%%%%%%

%%%%%%%%%%%%%%%%%%%%%%%%%%%%%%%%%%%%%%%%%
%%% BEGIN Q-HERDING APPLICATION BLOCK %%%
%%%%%%%%%%%%%%%%%%%%%%%%%%%%%%%%%%%%%%%%%
\begin{block}{\large Application: $q$-herding}


\end{block}
%%%%%%%%%%%%%%%%%%%%%%%%%%%%%%%%%%%%%%%
%%% END Q-HERDING APPLICATION BLOCK %%%
%%%%%%%%%%%%%%%%%%%%%%%%%%%%%%%%%%%%%%%

%%%%%%%%%%%%%%%%%%%%%%%%%%%%%%%%%%%
%%% BEGIN ACKNOWLEDGMENTS BLOCK %%%
%%%%%%%%%%%%%%%%%%%%%%%%%%%%%%%%%%%
\begin{block}{\large Acknowledgements}
We thank Cameron Freer for helpful conversations that led to this work. JHH was supported by the U.S. Government under FA9550-11-C-0028 and awarded by the Department of Defense, Air Force Office of Scientific Research, National Defense Science and Engineering Graduate (NDSEG) Fellowship, 32 CFR 168a.
\end{block}
%%%%%%%%%%%%%%%%%%%%%%%%%%%%%%%%%
%%% END ACKNOWLEDGMENTS BLOCK %%%
%%%%%%%%%%%%%%%%%%%%%%%%%%%%%%%%%

%%%%%%%%%%%%%%%%%%%%%%%%%%%%%%%%
%%% BEGIN BIBLIOGRAPHY BLOCK %%%
%%%%%%%%%%%%%%%%%%%%%%%%%%%%%%%%
\begin{block}{\small References}
\begin{footnotesize}
\bibliographystyle{plainnat}
\bibliography{herding}
\end{footnotesize}
\end{block}
%%%%%%%%%%%%%%%%%%%%%%%%%%%%%%
%%% END BIBLIOGRAPHY BLOCK %%%
%%%%%%%%%%%%%%%%%%%%%%%%%%%%%%


\end{column}

\end{columns}
\end{frame}
\end{document}