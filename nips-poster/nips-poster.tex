\documentclass[final]{beamer}
\mode<presentation>
{
 %\usetheme{Berlin}
 % \usetheme{Aachen}
%  \usetheme{Oldi6}
%  \usetheme{I6td}
 \usetheme{I6dv}
%  \usetheme{I6pd}
%  \usetheme{I6pd2}
}

\usepackage{bm}

\usepackage{times}
\usepackage{amsmath,amssymb,amsthm}

\usepackage[english]{babel}
\usepackage[latin1]{inputenc}

\usepackage{graphicx}
\usepackage{tabularx}
\usepackage{natbib}


\usepackage{hyperref,url}
\usepackage{amsmath,amssymb,amsthm}
\usepackage{tikz}
%\usepackage{float,subcaption,graphicx}
\usepackage{stmaryrd,wasysym,clrscode}
\usepackage{etex,etoolbox}
\usepackage{ifthen}
\usetikzlibrary{patterns,positioning}

\newcommand{\lc}[1]{#1_{\mathrm{loc}}}
\newcommand{\eq}[1]{\stackrel{\mathrm{#1}}{=}}
\DeclareMathOperator{\Var}{Var}
\DeclareMathOperator{\MMD}{MMD}
\newcommand{\MMDr}{\tilde{\MMD}}
\DeclareMathOperator{\Tr}{Tr}
\newcommand{\inner}[2]{\langle #1, #2 \rangle}
\newcommand{\E}{\mathcal{E}}
\newcommand{\eqdef}{\stackrel{\mathrm{def}}{=}}
\newcommand{\bP}{\mathbb{P}}
\newcommand{\bI}{\mathbb{I}}
\newcommand{\bE}{\mathbb{E}}
\newcommand{\sF}{\mathcal{F}}
\newcommand{\sC}{\mathcal{C}}
\newcommand{\C}{\mathcal{C}}
\newcommand{\sB}{\mathcal{B}}
\newcommand{\bR}{\mathbb{R}}
\newcommand{\sI}{\mathcal{I}}
\newcommand{\sP}{\mathcal{P}}
\newcommand{\sX}{\mathcal{X}}
\newcommand{\sS}{\mathcal{S}}
\newcommand{\sJ}{\mathcal{J}}
\newcommand{\sR}{\mathcal{R}}
\newcommand{\sN}{\mathcal{N}}
\newcommand{\meet}{\wedge}
\newcommand{\RE}[2]{\operatorname{RE}\left(#1 \ \| \ #2\right)}
\newcommand{\KL}[2]{\operatorname{KL}\left(#1 \ \| \ #2\right)}
\newcommand{\KLm}[2]{\operatorname{KL}_m\left(#1 \ \| \ #2\right)}
\newcommand{\score}[2]{\operatorname{score}\left(#1 \ \| \ #2\right)}
\newcommand{\phih}{\hat{\phi}}
\newcommand{\psih}{\hat{\psi}}
\DeclareMathOperator{\supp}{supp}
\DeclareMathOperator{\loc}{loc}
\DeclareMathOperator{\lub}{lub}
\newcommand{\atom}[1]{#1^{\circ}}
\newcommand{\stitch}[2]{\overline{#1}^{#2}}
\DeclareMathOperator{\argmin}{argmin}
\DeclareMathOperator{\argmax}{argmax}

\newtheorem{theorem}{Theorem}[section]
\newtheorem{lemma}[theorem]{Lemma}
\newtheorem{proposition}[theorem]{Proposition}
\newtheorem{corollary}[theorem]{Corollary}
\newtheorem{assumption}[theorem]{Assumption}
\theoremstyle{definition}
\newtheorem{example}[theorem]{Example}
\newtheorem{definition}[theorem]{Definition}
\newtheorem{remark}[theorem]{Remark}
\newtheorem{property}[theorem]{Property}

\def\ci{\perp\!\!\!\perp}


%\usepackage[orientation=landscape,size=a0,scale=1.4]{beamerposter}
\usepackage[orientation=landscape,size=custom,width=121.92,height=91.44,scale=1.4]{beamerposter} % 48in x 36in

% Display a grid to help align images
%\beamertemplategridbackground[1cm]

\title%[] %(optional, use only with long paper titles)
{A Greedy Framework for First-Order Optimization}

%\subtitle {Include Only If Paper Has a Subtitle}

\author%[Author, Another] % (optional, use only with lots of authors)
{Jonathan Huggins$^{*1}$, Jacob Steinhardt$^{*2}$}

\institute[] % (optional, but mostly needed)
{$^{*}$Both authors contributed equally to this work. $^1$Massachusetts Institute of Technology. $^{2}$Stanford University.}

\date{December 2013}

\begin{document}

\begin{frame}{} 
%\titlepage  
%\vfill
\begin{columns}
\begin{column}{0.48\linewidth}

%%%%%%%%%%%%%%%%%%%%%%%%%%%%%%%%%%%%%%%%%%%%%%%%%%
%%%% COLUMN 1		 						%%%%%%
%%%%%%%%%%%%%%%%%%%%%%%%%%%%%%%%%%%%%%%%%%%%%%%%%%

%%%%%%%%%%%%%%%%%%%%%%%%%%%%%%
%%% BEGIN MOTIVATION BLOCK %%%
%%%%%%%%%%%%%%%%%%%%%%%%%%%%%%
\begin{block}{\large Motivation}
Suppose we want to solve the following \emph{saddle point} problem:
\[ \min_u \max_{\theta} L(u, \theta), \]
where $L(u,\theta) = h(u) + \langle u, \theta \rangle - R(\theta)$. 
We assume that $h$ and $R$ are both convex and that $\langle \cdot, \cdot \rangle$ 
is an arbitrary bilinear map.

\textbf{Intuition:} $u$ is the ``primal'', $\theta$ is the ``dual''.

Tie-in with optimization: can recast as minimizing
\[ L(u) \eqdef \max_{\theta} L(u, \theta) = h(u) + R^*(u). \]

Here $R^*(u) \eqdef \max_{\theta} u^T\theta - R(\theta)$ is the 
\emph{Fenchel conjugate} of $\theta$.
\end{block}

\begin{block}{\large Fenchel conjugate properties}
Fenchel conjugates are a key component of \emph{Legendre-Fenchel duality} 
and also show up in mirror descent. It turns out that they will also be 
important for generalizing the Frank-Wolfe algorithm.

\textbf{Property 1:} If $R$ is convex then $R^{**} = R$.

\textbf{Property 2:} $\partial R^*(u) = \argmin_{\theta} u^T\theta - R(\theta)$.
\end{block}
%%%%%%%%%%%%%%%%%%%%%%%%%%%%
%%% END MOTIVATION BLOCK %%%
%%%%%%%%%%%%%%%%%%%%%%%%%%%%

%%%%%%%%%%%%%%%%%%%%%%%%%%%%%%
%%% BEGIN TOO GREEDY BLOCK %%%
%%%%%%%%%%%%%%%%%%%%%%%%%%%%%%
\begin{block}{\large Being (Too) Greedy}
JHH
\end{block}
%%%%%%%%%%%%%%%%%%%%%%%%%%%%
%%% END TOO GREEDY BLOCK %%%
%%%%%%%%%%%%%%%%%%%%%%%%%%%%

%%%%%%%%%%%%%%%%%%%%%%%%%%%%%
%%% BEGIN ALGORITHM BLOCK %%%
%%%%%%%%%%%%%%%%%%%%%%%%%%%%%
\begin{block}{\large Being Greedy Enough: Boosted Mirror Descent}
JHH

This block would contain the statement of the algorithms and the geometric intuition 
\end{block}
%%%%%%%%%%%%%%%%%%%%%%%%%%%
%%% END ALGORITHM BLOCK %%%
%%%%%%%%%%%%%%%%%%%%%%%%%%%
\end{column}	

	

%%%%%%%%%%%%%%%%%%%%%%%%%%%%%%%%%%%%%%%%%%%%%%%%%%
%%%% COLUMN 2		 						%%%%%%
%%%%%%%%%%%%%%%%%%%%%%%%%%%%%%%%%%%%%%%%%%%%%%%%%%

\begin{column}{0.48\linewidth}

%%%%%%%%%%%%%%%%%%%%%%%%%%%%%%%%
%%% BEGIN APPLICATIONS BLOCK %%%
%%%%%%%%%%%%%%%%%%%%%%%%%%%%%%%%
\begin{block}{\large Special Cases and Applications}
Many particular algorithms can be cast into our framework.
\begin{columns}[t]
\begin{column}{0.45\linewidth}

\begin{block}{Conditional gradient (Frank-Wolfe)}
Let $h \equiv 0$. Then we have the updates
\begin{align*}
u_t &= \argmin_u u^T\theta_t \\
 &= \argmin_u u^T\partial R^*(\hat{u}_{t-1}),
\end{align*}
which is the Frank-Wolfe update for $R^*$.
\end{block}

\begin{block}{Thresholded Frank-Wolfe}
Suppose we want to minimize $\|u\|_1 + R^*(u)$. Then 
we can consider the following 
\emph{thresholded Frank-Wolfe algorithm}:
\begin{align*}
u_t &= \argmin_u \|u\|_1 + u^T\partial R^*(\hat{u}_{t-1}),
\end{align*}
which converges at the same rate as conditional gradient 
but also imposes sparsity.
\end{block}

\begin{block}{(Modified) subgradient}
Let $h(u) = \frac{\gamma}{2} \|u\|_2^2$. Then the primal version of the 
algorithm gives the updates:
\begin{align*}
u_{t+1} &= \argmin_{\theta} \frac{\gamma}{2}\|u\|_2^2 + u^T\hat{\theta}_t \\
 &= -\frac{1}{\gamma}\hat{\theta}_t \\
 &= -\frac{1}{\gamma} \frac{\sum_{s=1}^t \alpha_s \partial R^*(u_s)}{\sum_{s=1}^t \alpha_s}.
\end{align*}
\end{block}

\end{column}
\begin{column}{0.45\linewidth}
\begin{block}{AHK low-rank SDP solver}
Suppose we want to solve the SDP
\begin{align*}
\text{maximize }   & \Tr(A^TX) \\
\text{subject to } & X \succeq 0 \\
                  & X_{ii} \leq 1 \, \forall i
\end{align*}
Set \small{$L(X,y) = \Tr(A^TX) + \sum_{i=1}^n \left[y_i(X_{ii}-1) - \eta y_i\log y_i\right]$}.
Then updating $X$ is an eigenvalue problem and each update is rank $1$. This 
yields a variant of the Arora-Hazan-Kale fast SDP solver.
\end{block}

\begin{block}{$q$-herding}
The \emph{herding} algorithm finds a distribution $\mu$ over 
$\sX$ such that $\bE_{x \sim \mu}[\phi(x)] \approx \bar{\phi}$ 
for a given collection of moments $\phi$. It is equivalent to 
the dual version of our algorithm for
\[ L(\mu,\theta) = \bE_{x \sim \mu}\left[\theta^T[\phi(x)-\bar{\phi}]\right] - \frac{1}{2}\|\theta\|_2^2, \]
which hinges upon $\|\bar{\phi}\|_2$ existing. In infinite 
dimensions this may not be the case and we can instead use the 
\emph{$q$-herding} algorithm based on the updates
\[ L(\mu,\theta) = \bE_{x \sim \mu}\left[\theta^T[\phi(x)-\bar{\phi}]\right] - \frac{1}{q}\|\theta\|_q^q, \]
which only requires that $\|\bar{\phi}\|_p$ exists, where $\frac{1}{p} + \frac{1}{q} = 1$.
\end{block}

\end{column}
\end{columns}
\end{block}
%%%%%%%%%%%%%%%%%%%%%%%%%%%%%%
%%% END APPLICATIONS BLOCK %%%
%%%%%%%%%%%%%%%%%%%%%%%%%%%%%%

%%%%%%%%%%%%%%%%%%%%%%%%%%%%%%%
%%% BEGIN CONVERGENCE BLOCK %%%
%%%%%%%%%%%%%%%%%%%%%%%%%%%%%%%
\begin{block}{\large Convergence of Boosted Mirror Descent}
\begin{columns}[t]
\begin{column}{0.45\linewidth}
\begin{theorem}
Consider the primal (respectively the dual) algorithm. Suppose that $h$ ($R$) is strongly convex with respect to a norm $\|\cdot\|$ 
and let $r = \sup_{\theta} \|\theta\|_{*}$ ($r = \sup_{u} \|u\|_{*}$). Then
\[ \sup_{\theta} L(\hat{u}, \theta) \leq \sup_{\theta} L(u^*, \theta) + \frac{2r^2}{A_T} \sum_{t=1}^T \frac{\alpha_{t+1}^2A_t}{A_{t+1}^2}. \]
\end{theorem}
\end{column}
\begin{column}{0.45\linewidth}
\begin{corollary} 
Under the hypotheses of the Theorem, for $\alpha_{t} = 1$ we have
\[ \sup_{\theta} L(\hat{u}, \theta) \leq \sup_{\theta} L(u^*, \theta) + \frac{2r^2 (\log (T) + 1)}{T}. \]
and for $\alpha_t = t$ we have
\[ \sup_{\theta} L(\hat{u}, \theta) \leq \sup_{\theta} L(u^*, \theta) + \frac{8r^2}{T}. \]
\end{corollary}
\end{column}
\end{columns}
\end{block}
%%%%%%%%%%%%%%%%%%%%%%%%%%%%%
%%% END CONVERGENCE BLOCK %%%
%%%%%%%%%%%%%%%%%%%%%%%%%%%%%

%%%%%%%%%%%%%%%%%%%%%%%%%%%%%%%%%%%%
%%%% BEGIN AHK APPLICATION BLOCK %%%
%%%%%%%%%%%%%%%%%%%%%%%%%%%%%%%%%%%%
%\begin{block}{\large Application: Solving SDPs}
%
%
%\end{block}
%%%%%%%%%%%%%%%%%%%%%%%%%%%%%%%%%%
%%%% END AHK APPLICATION BLOCK %%%
%%%%%%%%%%%%%%%%%%%%%%%%%%%%%%%%%%
%
%%%%%%%%%%%%%%%%%%%%%%%%%%%%%%%%%%%%%%%%%%
%%%% BEGIN Q-HERDING APPLICATION BLOCK %%%
%%%%%%%%%%%%%%%%%%%%%%%%%%%%%%%%%%%%%%%%%%
%\begin{block}{\large Application: $q$-herding}
%
%
%\end{block}
%%%%%%%%%%%%%%%%%%%%%%%%%%%%%%%%%%%%%%%%
%%%% END Q-HERDING APPLICATION BLOCK %%%
%%%%%%%%%%%%%%%%%%%%%%%%%%%%%%%%%%%%%%%%

%%%%%%%%%%%%%%%%%%%%%%%%%%%%%%%%%%%
%%% BEGIN ACKNOWLEDGMENTS BLOCK %%%
%%%%%%%%%%%%%%%%%%%%%%%%%%%%%%%%%%%
\begin{block}{\small Acknowledgements}
{\footnotesize Thanks to Cameron Freer, Simon Lacoste-Julien, Martin Jaggi, Percy Liang, Arun Chaganty, and Sida Wang 
for helpful discussions. Thanks also to the anonymous reviewer for suggesting improvements 
to the paper. JS was supported by the Hertz Foundation. JHH was supported by the U.S. Government under FA9550-11-C-0028 and awarded by the Department of Defense, Air Force Office of Scientific Research, National Defense Science and Engineering Graduate (NDSEG) Fellowship, 32 CFR 168a.}
\end{block}
%%%%%%%%%%%%%%%%%%%%%%%%%%%%%%%%%
%%% END ACKNOWLEDGMENTS BLOCK %%%
%%%%%%%%%%%%%%%%%%%%%%%%%%%%%%%%%

%%%%%%%%%%%%%%%%%%%%%%%%%%%%%%%%
%%% BEGIN BIBLIOGRAPHY BLOCK %%%
%%%%%%%%%%%%%%%%%%%%%%%%%%%%%%%%
\begin{block}{\small References}
\begin{footnotesize}
\bibliographystyle{plainnat}
\bibliography{herding}
\end{footnotesize}
\end{block}
%%%%%%%%%%%%%%%%%%%%%%%%%%%%%%
%%% END BIBLIOGRAPHY BLOCK %%%
%%%%%%%%%%%%%%%%%%%%%%%%%%%%%%


\end{column}

\end{columns}
\end{frame}
\end{document}
