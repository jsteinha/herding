\documentclass{article}
\usepackage{import, subfiles}
\usepackage{hyperref, icml2014, times}
\usepackage{tabularx}
\usepackage{algorithm}
\usepackage{algorithmic}
\usepackage{hyperref,url}
\usepackage{amsmath,amssymb,amsthm}
\usepackage{tikz}
%\usepackage{float,subcaption,graphicx}
\usepackage{stmaryrd,wasysym,clrscode}
\usepackage{etex,etoolbox}
\usepackage{ifthen}
\usetikzlibrary{patterns,positioning}

\newcommand{\lc}[1]{#1_{\mathrm{loc}}}
\newcommand{\eq}[1]{\stackrel{\mathrm{#1}}{=}}
\DeclareMathOperator{\Var}{Var}
\DeclareMathOperator{\MMD}{MMD}
\newcommand{\MMDr}{\tilde{\MMD}}
\DeclareMathOperator{\Tr}{Tr}
\newcommand{\inner}[2]{\langle #1, #2 \rangle}
\newcommand{\E}{\mathcal{E}}
\newcommand{\eqdef}{\stackrel{\mathrm{def}}{=}}
\newcommand{\bP}{\mathbb{P}}
\newcommand{\bI}{\mathbb{I}}
\newcommand{\bE}{\mathbb{E}}
\newcommand{\sF}{\mathcal{F}}
\newcommand{\sC}{\mathcal{C}}
\newcommand{\C}{\mathcal{C}}
\newcommand{\sB}{\mathcal{B}}
\newcommand{\bR}{\mathbb{R}}
\newcommand{\sI}{\mathcal{I}}
\newcommand{\sP}{\mathcal{P}}
\newcommand{\sX}{\mathcal{X}}
\newcommand{\sS}{\mathcal{S}}
\newcommand{\sJ}{\mathcal{J}}
\newcommand{\sR}{\mathcal{R}}
\newcommand{\sN}{\mathcal{N}}
\newcommand{\meet}{\wedge}
\newcommand{\RE}[2]{\operatorname{RE}\left(#1 \ \| \ #2\right)}
\newcommand{\KL}[2]{\operatorname{KL}\left(#1 \ \| \ #2\right)}
\newcommand{\KLm}[2]{\operatorname{KL}_m\left(#1 \ \| \ #2\right)}
\newcommand{\score}[2]{\operatorname{score}\left(#1 \ \| \ #2\right)}
\newcommand{\phih}{\hat{\phi}}
\newcommand{\psih}{\hat{\psi}}
\DeclareMathOperator{\supp}{supp}
\DeclareMathOperator{\loc}{loc}
\DeclareMathOperator{\lub}{lub}
\newcommand{\atom}[1]{#1^{\circ}}
\newcommand{\stitch}[2]{\overline{#1}^{#2}}
\DeclareMathOperator{\argmin}{argmin}
\DeclareMathOperator{\argmax}{argmax}

\newtheorem{theorem}{Theorem}[section]
\newtheorem{lemma}[theorem]{Lemma}
\newtheorem{proposition}[theorem]{Proposition}
\newtheorem{corollary}[theorem]{Corollary}
\newtheorem{assumption}[theorem]{Assumption}
\theoremstyle{definition}
\newtheorem{example}[theorem]{Example}
\newtheorem{definition}[theorem]{Definition}
\newtheorem{remark}[theorem]{Remark}
\newtheorem{property}[theorem]{Property}

\def\ci{\perp\!\!\!\perp}


%\allowdisplaybreaks

\icmltitlerunning{Adaptive Exponentiated Gradient}

\begin{document} 

\twocolumn[
\icmltitle{Adaptive Exponentiated Gradient}

\icmlauthor{Jacob Steinhardt}{jsteinhardt@cs.stanford.edu}
\icmladdress{Stanford University,
             353 Serra Street, Stanford, CA 94305 USA}
\icmlauthor{Jonathan Huggins}{jhuggins@mit.edu}
\icmladdress{MIT CSAIL,
             77 Massachusetts Avenue, Cambridge, MA 02139 USA}
\icmlauthor{Percy Liang}{pliang@cs.stanford.edu}
\icmladdress{Stanford University,
             353 Serra Street, Stanford, CA 94305 USA}


% You may provide any keywords that you 
% find helpful for describing your paper; these are used to populate 
% the "keywords" metadata in the PDF but will not be shown in the document
\icmlkeywords{online learning, optimistic, adaptive, matrix multiplicative weights}

\vskip 0.3in
]

\begin{abstract} 
We present an adaptive variant of the exponentiated gradient algorithm, 
achieving regret that depends only on the variance of the best retrospective 
expert. This generalizes and improves a result by 
\cite{hazan2010variation}. Our analysis is conceptually clean and brings together 
two recent developments in online learning, as well as a novel construction 
using an auxiliary instance of follow the regularized leader to tune a regret 
bound. Using the machinery developed, we are also able to extend our results 
to the matrix case, presenting an \emph{adaptive matrix exponentiated gradient} 
algorithm. Our proof in this case involves a novel analysis tool generalizing 
follow the regularized leader to vector-valued payoffs, which may be of 
independent interest.
\end{abstract} 

\subfile{introduction.tex}
\subfile{mw12.tex}
\subfile{machinery.tex}
\subfile{ftrl-aux.tex}
%\subfile{algorithms-table.tex}
\subfile{matrix.tex}
\subfile{conclusion.tex}

\bibliography{adaptive}
\bibliographystyle{icml2014}

\end{document} 
