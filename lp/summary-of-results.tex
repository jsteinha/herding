\documentclass[11pt]{article}
\usepackage{import}
\usepackage{hyperref,url}
\usepackage{amsmath,amssymb,amsthm}
\usepackage{tikz}
%\usepackage{float,subcaption,graphicx}
\usepackage{stmaryrd,wasysym,clrscode}
\usepackage{etex,etoolbox}
\usepackage{ifthen}
\usetikzlibrary{patterns,positioning}

\newcommand{\lc}[1]{#1_{\mathrm{loc}}}
\newcommand{\eq}[1]{\stackrel{\mathrm{#1}}{=}}
\DeclareMathOperator{\Var}{Var}
\DeclareMathOperator{\MMD}{MMD}
\newcommand{\MMDr}{\tilde{\MMD}}
\DeclareMathOperator{\Tr}{Tr}
\newcommand{\inner}[2]{\langle #1, #2 \rangle}
\newcommand{\E}{\mathcal{E}}
\newcommand{\eqdef}{\stackrel{\mathrm{def}}{=}}
\newcommand{\bP}{\mathbb{P}}
\newcommand{\bI}{\mathbb{I}}
\newcommand{\bE}{\mathbb{E}}
\newcommand{\sF}{\mathcal{F}}
\newcommand{\sC}{\mathcal{C}}
\newcommand{\C}{\mathcal{C}}
\newcommand{\sB}{\mathcal{B}}
\newcommand{\bR}{\mathbb{R}}
\newcommand{\sI}{\mathcal{I}}
\newcommand{\sP}{\mathcal{P}}
\newcommand{\sX}{\mathcal{X}}
\newcommand{\sS}{\mathcal{S}}
\newcommand{\sJ}{\mathcal{J}}
\newcommand{\sR}{\mathcal{R}}
\newcommand{\sN}{\mathcal{N}}
\newcommand{\meet}{\wedge}
\newcommand{\RE}[2]{\operatorname{RE}\left(#1 \ \| \ #2\right)}
\newcommand{\KL}[2]{\operatorname{KL}\left(#1 \ \| \ #2\right)}
\newcommand{\KLm}[2]{\operatorname{KL}_m\left(#1 \ \| \ #2\right)}
\newcommand{\score}[2]{\operatorname{score}\left(#1 \ \| \ #2\right)}
\newcommand{\phih}{\hat{\phi}}
\newcommand{\psih}{\hat{\psi}}
\DeclareMathOperator{\supp}{supp}
\DeclareMathOperator{\loc}{loc}
\DeclareMathOperator{\lub}{lub}
\newcommand{\atom}[1]{#1^{\circ}}
\newcommand{\stitch}[2]{\overline{#1}^{#2}}
\DeclareMathOperator{\argmin}{argmin}
\DeclareMathOperator{\argmax}{argmax}

\newtheorem{theorem}{Theorem}[section]
\newtheorem{lemma}[theorem]{Lemma}
\newtheorem{proposition}[theorem]{Proposition}
\newtheorem{corollary}[theorem]{Corollary}
\newtheorem{assumption}[theorem]{Assumption}
\theoremstyle{definition}
\newtheorem{example}[theorem]{Example}
\newtheorem{definition}[theorem]{Definition}
\newtheorem{remark}[theorem]{Remark}
\newtheorem{property}[theorem]{Property}

\def\ci{\perp\!\!\!\perp}

\DeclareMathOperator{\Regret}{Regret}
\usepackage{fullpage}

\begin{document}
\section{Summary of Results}
This is a summary of the main results for the upcoming ICML paper. I'm writing them here because it 
will hopefully help us organize the paper and we can also brainstorm the best way to present the results.

The first result is the adaptive mirror descent result from before, with the ability to add in 
``hints'' $s_t$ that guess what the next value of $z_t$ will be (and decrease the loss if the guess is 
accurate). This is important because it will later let us achieve bounds based on the variance of the $z_t$ 
rather than the second moment.
\begin{theorem}
For any sequence $s_1, s_2, \ldots, s_T$, consider the following prediction algorithm (we assume 
that $\|z_t\|_{\infty} \leq 1$ and $\eta \leq \frac{1}{2}$):
\begin{enumerate}
\item Predict with $w_{t,i} \propto \exp(x_{t,i} - \eta s_{t,i})$.
\item Receive loss $w_t^{\top}z_t$.
\item Update $x_{t+1,i} = x_{t,i} - \eta z_{t,i} - \eta^2 (z_{t,i} - s_{t,i})^2$.
\end{enumerate}
Then the regret relative to a fixed weight vector $w$ is bounded by
\[ \frac{\log(n)}{\eta} + \eta \sum_{i=1}^n w_i \sum_{t=1}^T (z_{t,i}-s_{t,i})^2. \]
\end{theorem}

Our major corollary uses a particular choice of $s_t$ to get bounds in terms of the variance. Interestingly, 
the proof technique involves analyzing the $s_t$ as the moves played by a FTRL-style algorithm, so we can 
view this algorithm as ``doubly-adaptive''.

Let $\Var_i \eqdef \sum_{t=1}^T (z_{t,i}-\mu_i)^2$, with $\mu \eqdef \frac{1}{T} \sum_{t=1}^T z_t$.
\begin{corollary}
In the above setting, for the choice $s_t = \frac{1}{t} \sum_{s=1}^{t-1} z_s$, the 
regret relative to $w$ is bounded by
\[ \frac{\log(n)}{\eta} + \eta \sum_{i=1}^n w_i\Var_i + O(\log(T)). \]
\end{corollary}
This result is a slight strengthening of one of the main results in the Hazan-Kale paper I showed you today. 
The places where it is stronger are that the constant is much better (additive $\log(T)$ rather than 
multiplicative factor of $16$) and also the regret is just in terms of the variance of the best expert, 
whereas before it was a more complicated metric in terms of largest cumulative variance of best expert 
at any point in time.

These same results hold in the matrix case. Now we assume that $-I \preceq Z_t \preceq I$ 
and continue to assume that $\eta \leq \frac{1}{2}$:
\begin{theorem}
For a sequence $S_1, \ldots, S_T$ of symmetric matrices, consider the following 
prediction algorithm:
\begin{enumerate}
\item Predict $W_t \propto \exp(X_t - \eta S_t)$.
\item Receive loss $\Tr(W_t^{\top}Z_t)$.
\item Update $X_{t+1} = X_t - \eta Z_t - \eta^2 (Z_t-S_t)^2$.
\end{enumerate}
Then the regret relative to a fixed matrix $W$ is bounded by
\[ \frac{\log(n)}{\eta} + \eta \sum_{i=1}^n \Tr((Z_t-S_t)^{\top}W(Z_t-S_t)). \]
\end{theorem}
We now define $\Var$ to be the matrix $\sum_{t=1}^T (Z_t-M)^2$, with $M \eqdef \frac{1}{T} \sum_{t=1}^T Z_t$.
\begin{corollary}
In the matrix setting, for the cohice $S_t = \frac{1}{t} \sum_{s=1}^{t-1} Z_s$, the 
regret relative to $w$ is bounded by
\[ \frac{\log(n)}{\eta} + \eta \Tr(W^{\top}\Var) + O(\log(T)). \]
\end{corollary}
It is important to note that in all of these bounds, the $O(\log(T))$ term does not 
contain any hidden factors of $n$ or $\eta$. In fact we could take $O(\log(T)) = 8\log(T) + 2$.
As far as I'm aware, both the theorem and the corollary in the matrix case are new. But 
a lot of the work done on this is in the theory community and I haven't reviewed 
references on matrix multiplicative weights nearly as thoroughly as for online learning.

Finally, we show an interesting separation between our adaptive algorithm and vanilla mirror 
descent even in the case where $s_t = 0$ for all $t$. For simplicity, we restrict our attention to the vector 
setting (the result extends trivially to the matrix setting). We consider the ``quasi-realizable'' setting, 
where one of the experts has identically zero loss (i.e. $z_{t,i} = 0$ for all $t$ for some $i$), and all 
other experts have non-negative cumulative loss. This is different from the realizable setting because 
losses can be both positive and negative.

\begin{proposition}
In the quasi-realizable setting, for any fixed step size $\eta$ there 
exists a sequence of losses such that mirror descent suffers regret $\Omega(\sqrt{T})$. 
In contrast, for any quasi-realizable sequence of losses and step size of $\eta = \frac{1}{2}$, 
adaptive mirror descent has regret $O(1)$ (where the $O(1)$ suppresses a $\log(n)$ dependence 
on dimension). These same results also hold in the matrix setting.
\end{proposition}


\end{document}
