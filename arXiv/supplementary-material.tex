\documentclass[reqno,oneside,a4paper]{amsart}
%\usepackage{jhh-misc}
\usepackage{hyperref}
\usepackage{epsfig}
\usepackage{framed}
\usepackage{natbib}
\usepackage{subfig}
\usepackage{xspace}
\usepackage[normalem]{ulem}

\usepackage{hyperref,url}
\usepackage{amsmath,amssymb,amsthm}
\usepackage{tikz}
%\usepackage{float,subcaption,graphicx}
\usepackage{stmaryrd,wasysym,clrscode}
\usepackage{etex,etoolbox}
\usepackage{ifthen}
\usetikzlibrary{patterns,positioning}

\newcommand{\lc}[1]{#1_{\mathrm{loc}}}
\newcommand{\eq}[1]{\stackrel{\mathrm{#1}}{=}}
\DeclareMathOperator{\Var}{Var}
\DeclareMathOperator{\MMD}{MMD}
\newcommand{\MMDr}{\tilde{\MMD}}
\DeclareMathOperator{\Tr}{Tr}
\newcommand{\inner}[2]{\langle #1, #2 \rangle}
\newcommand{\E}{\mathcal{E}}
\newcommand{\eqdef}{\stackrel{\mathrm{def}}{=}}
\newcommand{\bP}{\mathbb{P}}
\newcommand{\bI}{\mathbb{I}}
\newcommand{\bE}{\mathbb{E}}
\newcommand{\sF}{\mathcal{F}}
\newcommand{\sC}{\mathcal{C}}
\newcommand{\C}{\mathcal{C}}
\newcommand{\sB}{\mathcal{B}}
\newcommand{\bR}{\mathbb{R}}
\newcommand{\sI}{\mathcal{I}}
\newcommand{\sP}{\mathcal{P}}
\newcommand{\sX}{\mathcal{X}}
\newcommand{\sS}{\mathcal{S}}
\newcommand{\sJ}{\mathcal{J}}
\newcommand{\sR}{\mathcal{R}}
\newcommand{\sN}{\mathcal{N}}
\newcommand{\meet}{\wedge}
\newcommand{\RE}[2]{\operatorname{RE}\left(#1 \ \| \ #2\right)}
\newcommand{\KL}[2]{\operatorname{KL}\left(#1 \ \| \ #2\right)}
\newcommand{\KLm}[2]{\operatorname{KL}_m\left(#1 \ \| \ #2\right)}
\newcommand{\score}[2]{\operatorname{score}\left(#1 \ \| \ #2\right)}
\newcommand{\phih}{\hat{\phi}}
\newcommand{\psih}{\hat{\psi}}
\DeclareMathOperator{\supp}{supp}
\DeclareMathOperator{\loc}{loc}
\DeclareMathOperator{\lub}{lub}
\newcommand{\atom}[1]{#1^{\circ}}
\newcommand{\stitch}[2]{\overline{#1}^{#2}}
\DeclareMathOperator{\argmin}{argmin}
\DeclareMathOperator{\argmax}{argmax}

\newtheorem{theorem}{Theorem}[section]
\newtheorem{lemma}[theorem]{Lemma}
\newtheorem{proposition}[theorem]{Proposition}
\newtheorem{corollary}[theorem]{Corollary}
\newtheorem{assumption}[theorem]{Assumption}
\theoremstyle{definition}
\newtheorem{example}[theorem]{Example}
\newtheorem{definition}[theorem]{Definition}
\newtheorem{remark}[theorem]{Remark}
\newtheorem{property}[theorem]{Property}

\def\ci{\perp\!\!\!\perp}


\begin{document}

\title{Supplementary Material for ``Herding and its Generalizations via Boosted Mirror Descent''} 
\author{}
\date{}

\maketitle

\appendix

\section{CGD, MD, and Boosted MD}

The \bmd algorithms are very closely related to \cgd and \md. \dual and \cgd are related as follows. \dual minimizes $g + R^{*}$ while \cgd minimizes $h^{*} + f^{*}$. Make the identifications $h^{*} = R^{*}$ and $f^{*} = g$ and note the conditions on the functions being optimized are equivalent. Then with $\rho_{t} = \alpha_{t+1}/A_{t+1}$, we can rewrite the \cgd updates as 
\[
x_{t} &= \partial R^{*}(-\bar y_{t})  \\
y_{t} &= \partial g^{*}(x_{t}) \\
\bar y_{t+1} &= A_{t+1}^{-1}\sum_{s \le t+1} \alpha_{t} y_{t},
\]
which is identical to the \dual if we make the substitutions $x_{t} \to -\theta_{t}$ and $y_{t} \to -u_{t}$. 

We similarly relate \primal and \md. \primal minimizes $g + R^{*}$ while \md minimizes $h + f$. Make the identifications $h = g$ and $f = R^{*}$ and note the conditions of $h$ and $g$ are the same. Letting $\rho_{t+1} = \alpha_{t}/A_{t}$, we can rewrite the \md updates as 
\[
y_{t} &= \partial R^{*}(x_{t}) \\
\bar y_{t} &= \partial g^{*}((1 - \rho_{t+1})\partial g(x_{t}) - \rho_{t+1}y_{t}) \\
x_{t+1} &= \partial g^{*}(-\bar y_{t}), 
\]
where $\bar y_{t}$ is the same as above. Making the substitutions $y_{t} \to \theta_{t}$ and $x_{t} \to u_{t}$ completes the identification of the two methods. 

Finally, note that if we write $\alpha_{t} = \rho_{t}/\prod_{s=1}^{t} (1 - \rho_{s})$, we have $A_{t} = \prod_{s=1}^{t} (1 - \rho_{s})$ and hence $\alpha_{t} / A_{t} = \rho_{t}$. Hence, up to a change in index, we have a bijection between the $\rho_{t}$ and the $\alpha_{t}$. 

\section{Proof Lemma used in $q$-Herding Convergence Proof}

Two simple equalities were used in the proof of the convergence rate of $q$-herding. We now provide proofs. Using the fact that $x^{n} - y^{n} = (x - y)\sum_{i=0}^{n-1}x^{i-1}y^{n-i-1}$, for the first inequality we have
\(
x^{p} - py^{p-1}x + (p-1)y^{p}
&= x^{p} - y^{p} - py^{p-1}x + py^{p} \\
&= (x - y)\sum_{r=0}^{p-2} (y^{r}x^{p-r-1} - y^{p-1}) \\
&= (x - y)^{2}\sum_{r=0}^{p-2}\sum_{s=0}^{p-r-2}x^{p-r-s-2}y^{r+s} \\
&\le (x - y)^{2}\sum_{r=0}^{p-2}\sum_{s=0}^{p-r-2}\max(x, y)^{p-2} \\
&= {p \choose 2}(x - y)^{2}\max(x, y)^{p-2}.
\)
For the second inequality we have
\(
(y-z)^{2}\max(x,y)^{p-2} 
&\le 4\max(y,z)^{2}\max(x,y)^{p-2} \\
&\le 4\max(x,y,z)^{p}.
\)

%\section{Proof of Key Lemma}
%
%We now prove the key lemma, which we restate here, used in the convergence proofs of the \bmd algorithms.  
%\begin{lemma}
%\label{lem:bregman}
%Let $z_1,\ldots,z_T$ be vectors and let $f(x)$ be a strictly convex 
%function. Define $\hat{z}_t$ to be $\frac{1}{A_t} \sum_{s=1}^t \alpha_s z_s$.
%
%Let $x_1,\ldots,x_T$ be chosen via $x_{t+1} = \arg\min_{x} f(x) + \langle \hat{z}_t, x\rangle$. 
%Then for any $x^*$ we have
%\begin{align*}
%\lefteqn{\frac{1}{A_T} \sum_{t=1}^T \{\alpha_t(f(x_t) + \langle z_t, x_t \rangle)\}} \\
%\phantom{+} &\leq f(x^*) + \langle \hat{z}_t, x^* \rangle + \frac{1}{A_T} \sum_{t=1}^T A_t D_{f}(x_t \| x_{t+1}). 
%\end{align*}
%\end{lemma}
%\begin{proof}
%First note that, if $x_0 = \arg\min f(x) + \langle z, x \rangle$, 
%then $\nabla f(x_0) = -z$.
%
%%\NA{JNS: In the following proof, (2) is by the update formula for $x_{t+1}$ together with the preceding observation.  (3) is applying (2). (4) and (5) are re-arranging the sum. (6) is definition of Bregman divergence. (7) is the observation again. (8) is the definition of $x_{T+1}$.}
%
%Now note that
%\begin{align}
%\alpha_{t}z_{t} 
%&= A_{t}\hat z_{t} - A_{t-1}\hat z_{t-1} \\
%&= - A_{t}\nabla f(x_{t+1}) + A_{t-1} \nabla f(x_{t}),
%\end{align}
%so we have
%\begin{align}
%\lefteqn{\sum_{t=1}^T \{\alpha_t(f(x_t) + \langle z_t, x_t \rangle)\}} \\
% &= \sum_{t=1}^T \{\alpha_t f(x_t) + \langle A_{t-1} \nabla f(x_t) - A_t \nabla f(x_{t+1}), x_t \rangle\} \\
% &= \sum_{t=1}^T \{\alpha_t f(x_t) - \langle A_{t} \nabla f(x_{t+1}), x_t-x_{t+1} \rangle\} \\
% & \quad - A_{T}\langle \nabla f(x_{T+1}), x_{T+1} \rangle
%\end{align}
%\begin{align}
% &= \sum_{t=1}^T \{A_t f(x_t) - \langle A_{t} \nabla f(x_{t+1}), x_t-x_{t+1} \rangle - A_t f(x_{t+1})\}  \\
% &\quad+ A_T(f(x_{T+1}) - \langle \nabla f(x_{T+1}), x_{T+1} \rangle) \nonumber \\
% &= \sum_{t=1}^T \{A_tD_f(x_t \| x_{t+1})\}  \\
% &\quad+ A_T(f(x_{T+1}) - \langle \nabla f(x_{T+1}), x_{T+1} \rangle) \nonumber \\
% &= \sum_{t=1}^T \{A_tD_f(x_t \| x_{t+1})\} + A_T(f(x_{T+1}) + \langle \hat{z}_T, x_{T+1} \rangle) \\
% &\leq \sum_{t=1}^T \{A_tD_f(x_t \| x_{t+1})\} + A_T(f(x^*) + \langle \hat{z}_T, x^* \rangle). 
%\end{align}
%Dividing both sides by $A_T$ completes the proof.
%\end{proof}



\end{document}

