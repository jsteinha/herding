\documentclass[paper.tex]{subfiles}
\begin{document}

\section{Conclusion} 
\label{sec:conclusion}

In this paper we have introduced the primal and dual boosted mirror descent algorithms in order to investigate the properties of herding and a number of novel classes of generalized herding algorithms. In particular, $\alpha_{t}$-weighted herding is very similar to standard herding, but the pseudosamples are not required to have equal weights. We also defined the $q$-herding algorithm, which minimizes the distance between the sample estimator and the target moments as measured by the $\ell^{p}$ norm (with $p$ the conjugate of $q$, $p \ge 2$). For $p = 2$, we get herding. But in infinite dimensional space, $q$-herding with $p > 2$ may some times be applied when standard herding is inapplicable. We hope that our results will provide a useful consolidation of the many known phenomena related to herding, as well as 
form the basis for more new pseudo-sampling algorithms such as $q$-herding.

%We have also presented a number of convergence bounds for various classes of herding-like algorithms. We have also shown the limitations of herding-like algorithms in infinite dimensional spaces. 


%Using \bmd we also place the convergence results of \citet{Chen:2010a} and \citet{Bach:2012a} into a single framework. Furthermore, we were able to extend those convergence results to certain classes of herding-like algorithms including $\alpha_{t}$-weighted herding and $q$-herding%, and \NA{chen} herding.

%Our results thus extend the applicability and flexibility of deterministic algorithms for generating pseudosamples. While not universally applicable, these deterministic algorithms offer an increasingly attractive alternative to random sampling for certain types of problems and model classes because the can offer faster convergence rates, avoid local optima, come with convergence guarantees without the ``with high probability'' condition, and are often simple to implement. 

\end{document}
