\documentclass[paper.tex]{subfiles}
\begin{document}

\section{Composite Mirror Descent and Sparsity}
\label{sec:algorithm}

Our sparsity result is algorithmic in nature, and relies on an algorithm called 
\emph{composite mirror descent} \cite{comid}. In composite mirror descent, we 
are given a sequence of functions $f_t(w) = g_t(w) + h_t(w)$, and the goal is to 
minimize
\[ \sum_{t=1}^T \alpha_t f_t(w), \]
for some known non-negative weights $\alpha_t$. We assume throughout that $f_t$, 
$g_t$, and $h_t$ are all convex. The composite mirror descent algorithm also 
depends on an auxiliary convex function $\psi(w)$:
\begin{algorithm}
\caption{The composite mirror descent algorithm.}
\label{alg:comid}
\begin{algorithmic}
\STATE {\bfseries Composite Mirror Descent}
\STATE Initialize $\theta_0 \gets 0$, $\psi_0 \gets \psi$
\FOR{$t = 1$ {\bfseries to} $T$}
  \STATE $\psi_t \gets \psi_{t-1} + \alpha_t h_t$
  \STATE $w_t \gets \argmin_{w \in \W} \psi_{t}(w) + \theta_{t-1}^{\top} w$
  \STATE $\theta_t \gets \theta_t + \alpha_t \nabla g_t(w_t)$
\ENDFOR
\STATE Output $w^* = \frac{\sum_{t=1}^T \alpha_t w_t}{\sum_{t=1}^T \alpha_t}$
\end{algorithmic}
\end{algorithm}

The implicit assumption of this algorithm is that 
$\argmin_{w \in \W} \psi_t(w) + \theta_{t-1}^{\top}w$ can be computed efficiently. 
Among other things, this will typically require that $h_t$ has some known parametric 
form (as opposed to just being provided as a black box).

% give example
As an example, suppose that $\psi(w) = h_t(w) = \frac{1}{2}\|w\|_2^2$. Then 
$\psi_t(w) = \frac{t+1}{2}\|w\|_2^2$, and 
$\argmin_{w \in \W} \psi_t(w) + \theta_{t-1}^{\top} w = -\frac{1}{t+1}\theta_{t-1}$.
There are also many other examples of functions for which the $\argmin$ can be 
computed efficiently; this is closely related to the theory of Fenchel duality, 
which we present in Section~\ref{sec:fenchel} together with additional examples.

% define strong convexity
The convergence rate of Algorithm~\ref{alg:comid} depends on the degree of 
convexity of the functions $\psi$ and $h_t$. We can measure this convexity 
using any norm $\|\cdot\|$ on $\W$.
\begin{definition}
A differentiable function $f$ is said to be \emph{$\rho$-strongly convex} with 
respect to a norm $\|\cdot\|$ if
\[ f(w) - f(w') - \nabla f(w')^{\top}(w - w') \geq \frac{\rho}{2}\|w-w'\|^2 \]
for all $w, w' \in \W$.
\end{definition}
When $\rho = 1$ we sometimes just say that $f$ is strongly convex with respect to $\|\cdot\|$.
If $\|\cdot\|$ is the $l^2$ norm and $f$ is twice-differentiable, then $f$ is $\rho$-strongly 
convex if and only if $\nabla^2 f(w) \succeq \rho I$ for all $w \in \W$. In particular, 
$f(w) = \frac{1}{2}\|w - w_0\|_2^2$ is $1$-strongly convex with respsect to $\|\cdot\|_2$ for 
any $w_0$, although this is not true for general norms (for instance, 
$\frac{1}{2}\|w-w_0\|_1^2$ is not strongly convex with respect to $\|\cdot\|_1$ for any value 
of $\rho$). See Section~\ref{sec:fenchel} for more details.

One additional notion we need is that of a dual norm.
\begin{definition}
Given a norm $\|\cdot\|$, the \emph{dual norm} $\|\cdot\|_*$ is defined as 
\[ \|\theta\|_* \eqdef \sup_{\|w\| \leq 1} \theta^{\top}w. \]
\end{definition}
In other words, the dual norm is exactly constructed that H\"{o}lder's 
inequality $\theta^{\top}w \leq \|\theta\|_* \|w\|$ holds. In particular, 
the dual norm of $\|\cdot\|_2$ is $\|\cdot\|_2$, and the dual norm of 
$\|\cdot\|_p$ is $\|\cdot\|_{\frac{p}{p-1}}$.

% provide convergence theorem
We are now ready to state the convergence rate of Algorithm~\ref{alg:comid}.
\begin{theorem}
\label{thm:comid}
Let $\|\cdot\|$ be any norm on $\W$. Suppose that $\psi$ is $\rho_0$-strongly convex 
with respect to $\|\cdot\|$, and that $h_t$ is $\rho_t$-strongly convex with respect 
to $\|\cdot\|$. Suppose that $\|\nabla g_t\|_* \leq r$ for all $t$. Let 
$A_t = \sum_{s=1}^t \alpha_s$ and $R_t = \rho_0 + \sum_{s=1}^t \alpha_s \rho_s$. 
Then, defining $f(w) \eqdef \frac{1}{A_T} \sum_{t=1}^{T} \alpha_t f_t(w)$, we have
\[ f(w^*) \leq \inf_{w} f(w) + \frac{r^2}{2A_T} \sum_{t=1}^T \frac{\alpha_t^2}{R_t}. \]
\end{theorem}
The proof is given in Appendix~\ref{apx:comid}.

To unpack Theorem~\ref{thm:comid}, let us consider several special cases.
\begin{corollary}
Suppose that $\psi \equiv 0$ and $h_t$ is $\rho$-strongly convex with respect to 
$\|\cdot\|$ for each $t$ for some fixed $\rho$. Then, for $\alpha_t = 1$, we have
\[ f(w^*) \leq \inf_w f(w) + \frac{r^2 \log(T+1)}{2\rho T}, \]
and for $\alpha_t = t$ we have
\[ f(w^*) \leq \inf_w f(w) + \frac{2r^2}{\rho (T+1)}. \]
\end{corollary}
\begin{proof}
For $\alpha_t = 1$, we have $A_T = T$ and $R_t = \rho t$. Then Theorem~\ref{thm:comid} 
yields a convergence rate of
\begin{align*}
f(w^*) &\leq \inf_w f(w) + \frac{r^2}{2T} \sum_{t=1}^T \frac{1}{\rho t} \\
 &\leq \inf_w f(w) + \frac{r^2 \log(T+1)}{2\rho T}.
\end{align*}
For $\alpha_t = t$, we have $A_T = \frac{T(T+1)}{2}$ and $R_t = \rho \frac{t(t+1)}{2}$. 
Then Theorem~\ref{thm:comid} yields a convergence rate of
\begin{align*}
f(w^*) &\leq \inf_w f(w) + \frac{r^2}{T(T+1)} \sum_{t=1}^T \frac{2t^2}{\rho t(t+1)} \\
 &\leq \inf_w f(w) + \frac{r^2}{T(T+1)} \sum_{t=1}^T \frac{2}{\rho} \\
 &= \inf_w f(w) + \frac{2r^2}{\rho (T+1)}.
\end{align*}
\end{proof}

\begin{corollary}
Suppose that $\psi$ is $\rho$-strongly convex with respect to $\|\cdot\|$, but make 
no assumptions on $h_t$ other than convexity. Then, for $\alpha_t = 1$, we have
\[ f(w^*) \leq \inf_w f(w) + \frac{r^2}{2\rho}. \]
\end{corollary}
\begin{proof}
We have $A_T = T$ and $R_t = \rho$. Then Theorem~\ref{thm:comid} yields
\begin{align*}
f(w^*) &\leq \inf_w f(w) + \frac{r^2}{2T} \sum_{t=1}^T \frac{1}{\rho} \\
 &= \inf_w f(w) + \frac{r^2}{2\rho}.
\end{align*}
\end{proof}

% give boosted version (cite Bach, Shai)
\subsection{Composite Mirror Descent and Sparsity}
\label{sec:boosted}
What relationship does Composite Mirror Descent have to sparsification? The answer lies
in a technique called \emph{boosting} which will allow us to cleverly pick the functions 
$f_t(w)$. Boosting is an ubiquitous idea spanning multiple literatures; % TODO provide citations
in our case, we suppose that we have fixed functions $h : \W \to \bR$ and 
$f : \V \times \W \to \bR$ such that:
\begin{enumerate}
\item $f(\cdot, w)$ is linear in $v$ for each $w \in \W$.
\item $f(v, \cdot) - h(\cdot)$ is convex in $w$ for each $v \in \V$.
\end{enumerate}
Let $v_t = \argmax_{v \in V} f(v, w_t)$. Then we will pick 
$h_t(w) = h(w)$ and $g_t(w) = f(v_t, w) - h(w)$. Note that, 
even though $g_t$ depends on $v_t$ and hence $w_t$, we can still 
run Algorithm~\ref{alg:comid}, since we do not need access to $g_t$ 
until \emph{after} selecting $w_t$. 

This leads to the following primal-dual algorithm:
\begin{algorithm}
\caption{Primal-dual Composite Mirror Descent}
\label{alg:boosted-comid}
\begin{algorithmic}
\STATE Initialize $\theta_0 \gets 0$, $\psi_0 \gets \psi$
\FOR{$t = 1$ {\bfseries to} $T$}
  \STATE $\psi_t \gets \psi_{t-1} + \alpha_t h$
  \STATE $w_t \gets \argmin_{w \in \W} \psi_t(w) + \theta_{t-1}^{\top} w$
  \STATE $v_t \gets \argmax_{v \in \V} f(v, w_t)$
  \STATE $\theta_t \gets \theta_t + \alpha_t \left(\frac{\partial f(v_t, w_t)}{\partial w_t} - \frac{\partial h(w_t)}{\partial w_t}\right)$
\ENDFOR
\STATE Output $(v^*, w^*) = \left(\frac{\sum_{t=1}^T \alpha_t v_t}{\sum_{t=1}^T \alpha_t}, \frac{\sum_{t=1}^T \alpha_t w_t}{\sum_{t=1}^T \alpha_t}\right)$
\end{algorithmic}
\end{algorithm}

% give convergence theorem for boosted version
Algorithm~\ref{alg:boosted-comid} enjoys the following primal-dual convergence rate:
\begin{theorem}
TODO
\end{theorem}

\subsection{Fenchel duality}
\label{sec:fenchel}
% Fenchel conjugate properties
% -definition
% -convexity-smoothness duality
% -examples: L2, entropy, p-norm, matrix entropy, log-det
%  -compute Fenchel conjugate and derivative
% -cite Kakade paper, maybe others?

\subsection{Relations to Other Algorithms}
\label{sec:derived}
% derive gradient descent as a special case
% also Frank-Wolfe

\end{document}
