%%%%%%%%%%%%%%%%%%%%%%%%%%%%%%%%%%%%%%%%%%%%%%%%%%%%%%%%%%%%%%%%%%
%%%%%%%% ICML 2013 EXAMPLE LATEX SUBMISSION FILE %%%%%%%%%%%%%%%%%
%%%%%%%%%%%%%%%%%%%%%%%%%%%%%%%%%%%%%%%%%%%%%%%%%%%%%%%%%%%%%%%%%%

\documentclass{article}
\usepackage{import, icml2013}
\usepackage{hyperref,url}
\usepackage{amsmath,amssymb,amsthm}
\usepackage{tikz}
%\usepackage{float,subcaption,graphicx}
\usepackage{stmaryrd,wasysym,clrscode}
\usepackage{etex,etoolbox}
\usepackage{ifthen}
\usetikzlibrary{patterns,positioning}

%%% Added by Jonathan %%%
\def\[#1\]{\begin{align}#1\end{align}}
\def\(#1\){\begin{align*}#1\end{align*}}
\newcommand{\ip}[2]{\left\langle #1, #2 \right\rangle}
\definecolor{NAColor}{rgb}{.75,0,.75}
\newcommand{\NA}[1]{\textcolor{NAColor}{($\star$) #1}}
\newcommand{\dee}{\mathrm{d}}
\newcommand{\algname}[1]{\textsc{\lowercase{#1}}}
\def\argmax{\operatornamewithlimits{arg\,max}}
\def\argmin{\operatornamewithlimits{arg\,min}}
\newcommand{\defined}{\ensuremath{\triangleq}}
\newcommand{\bprf}{\begin{proof}}
\newcommand{\eprf}{\end{proof}}
\newcommand{\blem}{\begin{lemma}}
\newcommand{\elem}{\end{lemma}}
\newcommand{\eps}{\epsilon}
%%% End added by Jonathan %%%

\newcommand{\lc}[1]{#1_{\mathrm{loc}}}
\newcommand{\eq}[1]{\stackrel{\mathrm{#1}}{=}}
\DeclareMathOperator{\Var}{Var}
\DeclareMathOperator{\sign}{sign}
\DeclareMathOperator{\MMD}{MMD}
\newcommand{\MMDr}{\tilde{\MMD}}
\DeclareMathOperator{\Tr}{Tr}
\newcommand{\inner}[2]{\langle #1, #2 \rangle}
\newcommand{\E}{\mathcal{E}}
\newcommand{\eqdef}{\stackrel{\mathrm{def}}{=}}
\newcommand{\bP}{\mathbb{P}}
\newcommand{\bI}{\mathbb{I}}
\newcommand{\bE}{\mathbb{E}}
\newcommand{\sF}{\mathcal{F}}
\newcommand{\sH}{\mathcal{H}}
\newcommand{\sC}{\mathcal{C}}
\newcommand{\sM}{\mathcal{M}}
\newcommand{\sE}{\mathcal{E}}
\newcommand{\C}{\mathcal{C}}
\newcommand{\sB}{\mathcal{B}}
\newcommand{\bR}{\mathbb{R}}
\newcommand{\bN}{\mathbb{N}}
\newcommand{\bZ}{\mathbb{Z}}
\newcommand{\sI}{\mathcal{I}}
\newcommand{\sP}{\mathcal{P}}
\newcommand{\sX}{\mathcal{X}}
\newcommand{\sS}{\mathcal{S}}
\newcommand{\sJ}{\mathcal{J}}
\newcommand{\sR}{\mathcal{R}}
\newcommand{\sN}{\mathcal{N}}
\newcommand{\meet}{\wedge}
\newcommand{\RE}[2]{\operatorname{RE}\left(#1 \ \| \ #2\right)}
\newcommand{\KL}[2]{\operatorname{KL}\left(#1 \ \| \ #2\right)}
\newcommand{\KLm}[2]{\operatorname{KL}_m\left(#1 \ \| \ #2\right)}
\newcommand{\score}[2]{\operatorname{score}\left(#1 \ \| \ #2\right)}
\newcommand{\phih}{\hat{\phi}}
\newcommand{\psih}{\hat{\psi}}
\DeclareMathOperator{\supp}{supp}
\DeclareMathOperator{\loc}{loc}
\DeclareMathOperator{\lub}{lub}
\newcommand{\atom}[1]{#1^{\circ}}
\newcommand{\stitch}[2]{\overline{#1}^{#2}}
%\DeclareMathOperator{\argmin}{argmin}
%\DeclareMathOperator{\argmax}{argmax}

\newtheorem{theorem}{Theorem}[section]
\newtheorem{lemma}[theorem]{Lemma}
\newtheorem{proposition}[theorem]{Proposition}
\newtheorem{corollary}[theorem]{Corollary}
\newtheorem{assumption}[theorem]{Assumption}
\theoremstyle{definition}
\newtheorem{example}[theorem]{Example}
\newtheorem{definition}[theorem]{Definition}
\newtheorem{remark}[theorem]{Remark}
\newtheorem{property}[theorem]{Property}

\def\ci{\perp\!\!\!\perp}


\icmltitlerunning{Boosted Mirror Descent}

\begin{document} 

\twocolumn[
\icmltitle{Boosted Mirror Descent, Frank-Wolfe, and Herding}

\icmlauthor{Jacob Steinhardt}{jsteinhardt@cs.stanford.edu}
\icmladdress{Stanford University,
             353 Serra Street, Stanford, CA 94305 USA}
\icmlauthor{Jonathan Huggins}{jhuggins@mit.edu}
\icmladdress{Massachusetts Institute of Technology,
             43 Vassar Street, Cambridge, MA 02139 USA}

% You may provide any keywords that you 
% find helpful for describing your paper; these are used to populate 
% the "keywords" metadata in the PDF but will not be shown in the document
\icmlkeywords{boosting, optimization, mirror descent, frank-wolfe, herding}

\vskip 0.3in
]

\begin{abstract} 
TODO
\end{abstract} 

\section{Introduction}
\label{sec:intro}

\section{Boosted Mirror Descent}
\label{sec:algorithm}

\NA{JHH: I'm thinking that from the beginning we should let $u$ be defined over a different space than $\theta$ and assume a linear map $u \mapsto \eta(u) \in \Theta$. Then write $\langle \theta, \eta(u) \rangle$ so the setup will align with the herding case, where $\eta(u) = \bE_{u}[\phi(x)]$. }

Consider the following loss function:
\begin{equation}
L(u) = h(u) + \max_{\theta \in \Theta} \left\{\langle \theta, u \rangle - R(\theta) \right\}.
\end{equation}
We can think of $h$ as a \emph{primal regularizer} and $R$ as a \emph{dual regularizer}. 

Examples:
\begin{itemize}
\item Let $h(u) = 0$ and $R(\theta) = \langle \theta, u_0 \rangle$. Then $L(u)$ is the 
      \emph{maximum mean discrepancy} between $u$ and $u_0$ 
      relative to $\Theta$.
\item Let $h(u) = 0$ and $R(\theta) = S^*(\theta)$ where $S^*$ is the Fenchel conjugate of 
      any strongly convex function $S$. Then $L(u) = S(u)$.
\item Let $h(u) = \|u\|_1$, $R(u) = \langle \theta, u_0 \rangle + \frac{1}{2} \|\theta\|_2^2$. 
      Then $L(u) = \|u\|_1 + \frac{1}{2} \|u-u_0\|_2^2$. 
\end{itemize}

In this paper we will consider a family of methods for minimizing $L(u)$, 
which are based on Bach's generalization of the Frank-Wolfe algorithm and can 
be interpreted as boosted mirror descent.

\NA{JHH: Should we comment on the symmetry introduced by this formulation?}
First, let us generalize the setting 
to consider a \emph{two-argument} loss function
\begin{equation}
L(u,\theta) = h(u) + \langle \theta, u \rangle - R(\theta).
\end{equation}
We will assume throughout that $h$ and $R$ are both convex functions, 
and furthermore that $\arg\max_{u} h(u) + \theta^Tu$ can be efficiently 
computed for all values of $\theta$.

Mirror descent, together with boosting, yields an algorithm for finding 
``saddle points'' of $L$. It is the following algorithm (called Algorithm 1).
For notational convenience, for a sequence of weights $\alpha_1,\alpha_2,\ldots$ 
let $\hat{u}_t = \frac{\sum_{s=1}^t \alpha_su_s}{\sum_{s=1}^t \alpha_s}$ and let 
$\hat{\theta}_t = \frac{\sum_{s=1}^t \alpha_s\theta_s}{\sum_{s=1}^t \alpha_s}$.
\begin{enumerate}
\item $u_1 \in \arg\min_u h(u)$
\item $\theta_{t} \in \arg\max_{\theta \in \Theta} \langle \theta, u_t \rangle - R(\theta)$
\item $u_{t+1} \in \arg\min_{u} h(u) + \langle \hat{\theta}_t, u \rangle$
%\item $u_{t+1} \in \arg\min_{u} h(u) + \langle \frac{1}{t} \sum_{s \leq t} \theta_s, u \rangle$
\end{enumerate}
As long as $h$ is strongly convex, we obtain the 
bound (see Proposition~\ref{cor:method-1}):
\begin{equation}
\sup_{\theta \in \Theta} L(\hat{u}_T, \theta) \leq \sup_{\theta \in \Theta} L(u^*, \theta) + O(1/T).
\end{equation}
In other words, $\hat{u}_T$ is close to being a global minimum of $L$.
However, it is often the case that $h$ is not strongly convex, whereas $R$ is strongly convex. In this case 
we may wish to use the following slightly different algorithm (called Algorithm 2):
\begin{enumerate}
\item $\theta_1 \in \arg\min_{\theta} R(\theta)$
\item $u_t \in \arg\min_{u} h(u) + \langle \theta_t, u \rangle$
%\item $\theta_{t+1} \in \arg\max_{\theta \in \Theta} \langle \theta, \frac{1}{t} \sum_{s \leq t} u_s \rangle - R(\theta)$
\item $\theta_{t+1} \in \arg\max_{\theta \in \Theta} \langle \theta, \hat{u}_t \rangle - R(\theta)$
\end{enumerate}
We then obtain the following bound 
when $R$ is strongly convex (see Proposition~\ref{cor:method-2}):
\[ \sup_{\theta \in \Theta} L(\hat{u}_T, \theta) \leq \sup_{\theta \in \Theta} L(u^*, \theta) + O(1/T). \]
Algorithm 1 and Algorithm 2 are closely related; indeed, they are dual to each other 
(performing Algorithm 1 on $L(u,\theta)$ is the same as performing Algorithm 2 on 
$-L(\theta,u)$).
\NA{JHH: It would be more elegant to argue either here or below that the bounds on the second algorithm follow directly from duality}
\NA{JNS: The statements aren't actually dual to each other; the first is saying that the primal converges to its 
    optimum, and the second is saying that the dual converges to its optimum.}

\section{Convergence Proofs}
\label{sec:proofs}
We now prove the convergence results cited in Section~\ref{sec:algorithm}. 
Throughout this section, assume that $\alpha_1,\ldots,\alpha_T$ is 
a sequence of real numbers and that $A_t = \sum_{s=1}^t \alpha_s$. 
We further require that $A_t \geq 0$ for all $t$.

Also recall that the Bregman divergence is defined by 
$D_f(x_2 \| x_1) \eqdef f(x_2) - \langle \nabla f(x_1), x_2-x_1 \rangle - f(x_1)$.

Our proofs hinge on the following key lemma:
\begin{lemma}
\label{lem:bregman}
Let $z_1,\ldots,z_T$ be vectors and let $f(x)$ be a strictly convex 
function. Define $\hat{z}_t$ to be $\frac{1}{A_t} \sum_{s=1}^t \alpha_s z_s$.

Let $x_1,\ldots,x_T$ be chosen via $x_{t+1} = \arg\min_{x} f(x) + \langle \hat{z}_t, x\rangle$. 
Then for any $x^*$ we have
\begin{align*}
\lefteqn{\frac{1}{A_T} \sum_{t=1}^T \{\alpha_t(f(x_t) + \langle z_t, x_t \rangle)\}} \\
\phantom{+} &\leq f(x^*) + \langle \hat{z}_t, x^* \rangle + \frac{1}{A_T} \sum_{t=1}^T A_t D_{f}(x_t \| x_{t+1}). 
\end{align*}
\end{lemma}
\begin{proof}
First note that, if $x_0 = \arg\min f(x) + \langle z, x \rangle$, 
then $\nabla f(x_0) = -z$.

\NA{JNS: In the following proof, (2) is by the update formula for $x_{t+1}$ together with the preceding observation.  (3) is applying (2). (4) and (5) are re-arranging the sum. (6) is definition of Bregman divergence. (7) is the observation again. 
(8) is the definition of $x_{T+1}$.}

Now note that
\begin{align}
\alpha_{t}z_{t} = A_{t}\hat z_{t} - A_{t-1}\hat z_{t-1} = - A_{t}\nabla f(x_{t+1}) + A_{t-1} \nabla f(x_{t}) 
\end{align}
so we have
\begin{align}
\lefteqn{\sum_{t=1}^T \{\alpha_t(f(x_t) + \langle z_t, x_t \rangle)\}} \\
 &= \sum_{t=1}^T \{\alpha_t f(x_t) + \langle A_{t-1} \nabla f(x_t) - A_t \nabla f(x_{t+1}), x_t \rangle\} \\
 &= \sum_{t=1}^T \{\alpha_t f(x_t) - \langle A_{t} \nabla f(x_{t+1}), x_t-x_{t+1} \rangle\} - A_{T}\langle \nabla f(x_{T+1}), x_{T+1} \rangle \\
 &= \sum_{t=1}^T \{A_t f(x_t) - \langle A_{t} \nabla f(x_{t+1}), x_t-x_{t+1} \rangle - A_t f(x_{t+1})\}  \\
 &\quad+ A_T(f(x_{T+1}) - \langle \nabla f(x_{T+1}), x_{T+1} \rangle) \nonumber \\
 &= \sum_{t=1}^T \{A_tD_f(x_t \| x_{t+1})\} + A_T(f(x_{T+1}) - \langle \nabla f(x_{T+1}), x_{T+1} \rangle) \\
 &= \sum_{t=1}^T \{A_tD_f(x_t \| x_{t+1})\} + A_T(f(x_{T+1}) + \langle \hat{z}_T, x_{T+1} \rangle) \\
 &\leq \sum_{t=1}^T \{A_tD_f(x_t \| x_{t+1})\} + A_T(f(x^*) + \langle \hat{z}_T, x^* \rangle). \\
\end{align}
Dividing both sides by $A_T$ completes the proof.
\end{proof}
We also note that $D_f(x_t \| x_{t+1}) = D_{f^*}(\hat{z}_{t+1} \| z_t)$, where $f^*(z) = \sup_x \langle z,x\rangle - f(x)$. 
This form of the bound will often be more useful to us.
\begin{lemma}
\label{lem:convexity}
Suppose that $D_f(x' \| x) \geq \frac{1}{2}\|x-x'\|^2$ for some 
norm $\|\cdot\|$. (In this case we say that $f$ is strongly 
convex with respect to $\|\cdot\|$.) 
Then $D_{f^*}(x' \| x) \leq \frac{1}{2}\|x-x'\|_{*}^2$.
\end{lemma}
\begin{proposition}[Convergence of Algorithm 1]
\label{prop:method-1}
Consider the updates $\theta_t \in \arg\max_{\theta} \langle \theta, u_t \rangle - R(\theta)$ 
and $u_{t+1} \in \arg\min_u h(u) + \langle \hat{\theta}_s, u \rangle$. 
Then we have
\begin{equation}
\sup_{\theta} L(\hat{u}_T, \theta) \leq \sup_{\theta} L(u^*, \theta) + \frac{1}{A_T} \sum_{t=1}^T A_tD_f(x_t \| x_{t+1}).
\end{equation}
\end{proposition}
\begin{proof}
Note that $L(u_t, \theta_t) = \max_{\theta} L(u_t, \theta)$ by construction. 
Also note that, if we invoke Lemma~\ref{lem:bregman} with $f = h$ and 
$z_t = \theta_t$, then we get the inequality
\begin{equation}
\frac{1}{A_T} \sum_{t=1}^T \alpha_t L(u_t, \theta_t) \leq \frac{1}{A_T} \sum_{t=1}^T \alpha_t L(u^*, \theta_t) + \frac{1}{A_T} \sum_{t=1}^T A_tD_f(x_t \| x_{t+1}).
\end{equation}
Combining these together, we get the string of inequalities
\begin{align*}
L(\hat{u}_T, \theta) &= L\left(\frac{1}{A_T} \sum_{t=1}^T \alpha_tu_t, \theta\right) \\
 &\leq \frac{1}{A_T} \sum_{t=1}^T \alpha_t L(u_t, \theta) \\
 &\leq \frac{1}{A_T} \sum_{t=1}^T \alpha_t L(u_t, \theta_t) \\
 &\leq \frac{1}{A_T} \sum_{t=1}^T \alpha_t L(u^*, \theta_t) + \frac{1}{A_T} \sum_{t=1}^T A_t D_f(u_t \| u_{t+1}) \\
 &\leq \sup_{\theta} L(u^*, \theta) + \frac{1}{A_T} \sum_{t=1}^T A_tD_f(u_t \| u_{t+1}),
\end{align*}
as was to be shown.
\end{proof}
\begin{corollary}
\label{cor:method-1}
Suppose that $h$ is strongly convex with respect to a norm $\|\cdot\|$ 
and let $r = \sup_{\theta} \|\theta\|_{*}$. Then 
\[ \sup_{\theta} L(\hat{u}, \theta) \leq \sup_{\theta} L(u^*, \theta) + \frac{2r^2}{A_T} \sum_{t=1}^T \frac{\alpha_{t+1}^2A_t}{A_{t+1}^2}. \]
In particular, for $\alpha_t = t$, we have
\[ \sup_{\theta} L(\hat{u}, \theta) \leq \sup_{\theta} L(u^*, \theta) + \frac{8r^2}{T^2}. \]
\end{corollary}
\begin{proof}
By Lemma~\ref{lem:convexity}, we have 
\begin{align*}
D_f(x_t \| x_{t+1}) &= D_{f^*}(\hat{\theta}_{t+1} \| \hat{\theta}) \\
 &\leq \frac{1}{2}\|\theta_{t+1}-\theta\|_{*}^2 \\
 &= \frac{1}{2}\|\frac{\sum_{s \leq t}\alpha_s\theta_s}{\sum_{s \leq t} \alpha_s} - \frac{\sum_{s \leq t+1}\alpha_s\theta_s}{\sum_{s \leq t+1} \alpha_s}\|_{*}^2 \\
 &= \frac{1}{2}\|\frac{\alpha_{t+1}}{A_tA_{t+1}} \sum_{s \leq t} \alpha_s\theta_s - \frac{\alpha_{t+1}}{A_{t+1}} \theta_{t+1}\|_{*}^2 \\
 &\leq \frac{1}{2} \left(\frac{\alpha_{t+1}}{A_tA_{t+1}} \sum_{s \leq t} \alpha_s\|\theta_s\|_{*}^2 + \frac{\alpha_{t+1}}{A_{t+1}} \|\theta_{t+1}\|_{*}^2\right)^2 \\
 &\leq \frac{2r^2\alpha_{t+1}^2}{A_{t+1}^2}.
\end{align*}
It follows that 
\begin{align*}
\frac{1}{A_T} \sum_{t=1}^T A_tD_f(x_t \| x_{t+1}) &\leq \frac{2r^2}{A_T} \sum_{t=1}^T \frac{\alpha_{t+1}^2A_t}{A_{t+1}^2}.
\end{align*}
In particular, if we let $\alpha_t = t$, then $A_t = \frac{t(t+1)}{2}$ and 
$\frac{\alpha_{t+1}^2A_t}{A_{t+1}^2} = \frac{2(t+1)^2t(t+1)}{(t+1)^2(t+2)^2} = \frac{2t(t+1)}{(t+2)^2} \leq 2$.
We therefore get
\begin{equation}
\frac{2r^2}{A_T} \sum_{t=1}^T \frac{\alpha_{t+1}^2A_t}{A_{t+1}^2} \leq \frac{8r^2}{T(T+1)},
\end{equation}
which completes the proof.
\end{proof}
\begin{proposition}[Convergence of Algorithm 2]
\label{prop:method-2}
Consider the updates $u_t \in \arg\min_{u} h(u) + \langle \theta_t, u \rangle$ 
and $\theta_{t+1} \in \arg\max_{\theta} \langle \theta, \hat{u}_t \rangle - R(\theta)$. 
Then we have 
\begin{equation}
\sup_{\theta} L(\hat{u}, \theta) \leq \sup_{\theta} L(u^*, \theta) + \frac{1}{A_T} \sum_{t=1}^T A_tD_{R}(\theta_t \| \theta_{t+1}).
\end{equation}
\end{proposition}
\begin{proof}
If we invoke Lemma~\ref{lem:bregman} with $f=R$ and $z_t=-u_t$, then we get 
the inequality
\[ \frac{1}{A_T} \sum_{t=1}^T -\alpha_t L(u_t, \theta_t) \leq \frac{1}{A_T} \sum_{t=1}^T -\alpha_t L(u_t, \theta^*) - \frac{1}{A_T} \sum_{t=1}^T A_tD_R(\theta_t \| \theta_{t+1}). \]
Re-arranging yields
\[ \frac{1}{A_T }\sum_{t=1}^T \alpha_t L(u_t, \theta^*) \leq \frac{1}{A_T} \sum_{t=1}^T \alpha_t L(u_t, \theta_t) + \frac{1}{A_T} \sum_{t=1}^T A_tD_R(\theta_t \| \theta_{t+1}). \]
Now, we have the following string of inequalities:
\begin{align*}
L(\hat{u}, \theta) &= L\left(\frac{1}{A_T} \sum_{t=1}^T \alpha_t u_t, \theta\right) \\
 &\leq \frac{1}{A_T} \sum_{t=1}^T \alpha_t L(u_t, \theta) \\
 &\leq \frac{1}{A_T} \sum_{t=1}^T \alpha_t L(u_t, \theta_t) + \frac{1}{A_T} \sum_{t=1}^T A_t D_R(\theta_t \| \theta_{t+1}) \\
 &= \frac{1}{A_T} \sum_{t=1}^T \alpha_t \inf_{u} L(u, \theta_t) + \frac{1}{A_T} \sum_{t=1}^T A_tD_R(\theta_t \| \theta_{t+1}) \\
 &\leq \frac{1}{A_T} \sum_{t=1}^T \alpha_t L(u^*, \theta_t) + \frac{1}{A_T} \sum_{t=1}^T A_tD_R(\theta_t \| \theta_{t+1}) \\
 &\leq \frac{1}{A_T} \sum_{t=1}^T \alpha_t \sup_{\theta} L(u^*, \theta) + \frac{1}{A_T} \sum_{t=1}^T A_tD_R(\theta_t \| \theta_{t+1}) \\
 &= \sup_{\theta} L(u^*, \theta) + \frac{1}{A_T} \sum_{t=1}^T A_tD_R(\theta_t \| \theta_{t+1}),
\end{align*}
as was to be shown.
\end{proof}
\begin{corollary}
\label{cor:method-2}
Suppose that $R$ is strongly convex with respect to a norm $\|\cdot\|$ 
and let $r = \sup_{u} \|u\|_{*}$. Then 
\[ \sup_{\theta} L(\hat{u}, \theta) \leq \sup_{\theta} L(u^*, \theta) + \frac{2r^2}{A_T} \sum_{t=1}^T \frac{\alpha_{t+1}^2A_t}{A_{t+1}^2}. \]
In particular, for $\alpha_t = t$, we have
\[ \sup_{\theta} L(\hat{u}, \theta) \leq \sup_{\theta} L(u^*, \theta) + \frac{8r^2}{T^2}. \]
\end{corollary}
\begin{proof}
The proof is identical to the previous corollary.
\end{proof}

\section{Application to Herding}
\label{sec:herding}

\NA{JHH: Need to specify the spaces we are working with here.}
Suppose we are given a family of features $\phi$ together 
with a known value $\bar{\phi}$ for $\bE_{\mu}[\phi]$. We 
would like to construct a distribution that approximately 
matches this distribution, i.e.~one for which 
$\bE_{\hat{\mu}}[\phi] \approx \bar{\phi}$. A natural way 
to do this is by using boosted mirror descent to minimize the 
maximum mean discrepancy relative to $\phi$; 
in other words, to minimize
\[ L(\mu) = \max_{i \in \{1,\dots,d\}} |\bE_{\mu}[\phi_i]-\bar{\phi}_i|. \]
If we wish to write this more similarly to the original problem 
formulation, we can write
\[ L(\mu) = \sup_{\|\theta\|_1 \leq 1} \theta^T(\bE_{\mu}[\phi]-\bar{\phi}). \]
One issue is that $\|\theta\|_1$ is not strongly convex, so we relax $\|\theta\|_1$ 
to $\|\theta\|_2$ and write our loss function in the Lagrangian form:
\[ L(\mu, \theta) = \theta^T\bE_{\mu}[\phi] - \theta^T\bar{\phi} - \frac{1}{2}\|\theta\|_2^2. \]
Intriguingly, Algorithm 1 and Algorithm 2 are identical to each other for this choice of $L$ 
(up to a change of index). In this case, $\mu_t = \delta_{x_t}$ for some $x_t$, and 
the updates are (for Algorithm 2):
\begin{itemize}
\item $x_t \in \arg\min_{x} \theta_t^T\phi(x)$
\item $\theta_{t+1} = \frac{1}{A_t} \sum_{s=1}^t \alpha_s \phi(x)$.
\end{itemize}
These are the same as the standard herding updates when 
$\alpha_t = 1$. Note that $R$ is strongly convex with respect to 
$\|\cdot\|_2$, which is its own dual. Therefore, if we let
$r = \sup_{x} \|\phi(x)\|_2$, then Corollary~\ref{cor:method-2} gives us
\begin{align*}
\sup_{\theta} L\left(\frac{1}{T} \sum_{t=1}^T \delta_{x_t}, \theta\right) &\leq \inf_{u^*} \sup_{\theta} L(u^*, \theta) + \frac{2r^2\log(T+1)}{T} \\
 &= \sup_{\theta} \inf_{u^*} L(u^*, \theta) + \frac{2r^2\log(T+1)}{T} \\
 &= \frac{2r^2\log(T+1)}{T}.
\end{align*}
This is a slightly weaker version of the typical herding bound on MMD.
Note that if we use $\alpha_t = t$, which corresponds to using a weighted 
average, then we obtain a much stronger bound of $\frac{8r^2}{T^2}$ instead.

\section{Herding in Infinite Dimensions}
\label{sec:infinite-case}

\section{Herding and Maximum Entropy}
\label{sec:max-ent}

Bach et al. have observed that herding appears to be approximately building 
samples from the maximum entropy distribution for a given collection of moment 
constraints. In this section we will make this connection more explicit 
as well as draw other relationships between herding and maximum entropy.

Suppose that instead of setting $h(u) = 0$ we set $h(u)$ to a small multiple of 
the negative entropy: $h(u) = \eta \bE_{u}[\log u(x)]$. We will still use 
$R(\theta) = \frac{1}{2}\|\theta\|_2^2$. In this case, Algorithm 2 gives the updates
\begin{itemize}
\item $u_t(x) \propto \exp\left(-\frac{1}{\eta}\theta_t^T\phi(x)\right)$
\item $\theta_{t+1} = \bE_{\hat{u}_t}[\phi(x)]$.
\end{itemize}
Using Corollary~\ref{cor:method-2} gives us
\begin{align*}
\sup_{\theta} L\left(\frac{1}{T} \sum_{t=1}^T u_t, \theta\right) &\leq \inf_{u^*} \sup_{\theta} L(u^*, \theta) + \frac{2r^2\log(T+1)}{T} \\
 &= \inf_{u^*} \left\{-\eta H(u^*) + \frac{1}{2}\|\bE_{u^*}[\phi(x)]-\bar{\phi}\|_2^2\right\} \\ &\phantom{=} + \frac{2r^2\log(T+1)}{T}.
\end{align*}
In particular, let $u^*$ be the maximum entropy distribution satisfying $\bE_{u^*}[\phi(x)] = \bar{\phi}$, 
and let $H^*$ be the corresponding entropy. Then we have 
\[ -\eta H(\hat{u}) + \|\bE_{\hat{u}}[\phi(x)]-\bar{\phi}\|_2^2 \leq -\eta H^* + \frac{2r^2\log(T+1)}{T}. \]
Since $\|\cdot\|_2^2 \geq 0$, this implies that 
\begin{equation}
H(\hat{u}) \geq H^* - \frac{2r^2\log(T+1)}{\eta T},
\end{equation}
 and so asymptotically $\hat{u}$ has at least 
as much entropy as $u^*$.

Now, suppose that our state space has some finite size $S$, so 
that no distribution has entropy greater than $\log(S)$. Then 
we conclude that
\begin{equation}
\|\bE_{\hat{u}}[\phi(x)]-\bar{\phi}\|_2^2 \leq \eta[\log(S)-H^*] + \frac{2r^2\log(T+1)}{T}.
\end{equation}

If we let $\eta \to 0$ and $\eta T \to \infty$, then we obtain 
a distribution $\hat{u}$ whose moments are arbitrarily close to 
the moments of $u^*$, and whose entropy is almost as large as 
(or larger than) the entropy of $u^*$. 

% \begin{lemma}
% blah
% \end{lemma}
% \begin{proof}
% Let $L(u) = -H(u) + \sup_{\theta} \theta^T(\bE_{u}[\phi(x)]-\bar{\phi})$. 
% Clearly $L$ is minimized at the maximum entropy distribution $u^*$ satisfying 
% $\bE_{u}[\phi(x)] = \bar{\phi}$. Let $\theta^*$ be the natural parameters 
% for $u^*$, then we can lower bound $L(u)$ by $-H(u^*) + (\theta^*)^T(\bE_{u}[\phi(x)]-\bar{\phi})$.
% 
% Since $-\lambda H(u) + \|\bE_{u}[\phi(x)]-\bar{\phi}\|_2^2$ is convex, we have
% $-\lambda H(u) + \|\bE_{u}[\phi(x)]-\bar{\phi}\|_2^2 \geq -\lambda H(u_0) + \|\bE_{u_0}[\phi(x)]-\bar{\phi}\|_2^2 + (\bE_{u_0}[\phi(x)]-\bar{\phi})^T(u-u_0)$.
% Setting $u = u^*$ yields
% \[ -\lambda H(u_0) + \frac{1}{2} \|\bE_{u_0}[\phi(x)]-\bar{\phi}\|_2^2 + (\bE_{u_0}[\phi(x)]-\bar{\phi})^T(u-u_0) \leq -\lambda H(u^*). \]
% 
% \end{proof}

%% Let us now consider the entropic regularizer $h(u) = \tau\bE_{u}[\log u(x)]$ with 
%% \[ R(\theta) = \bE_{u}[\theta^T\bar{\phi}] + \left\{ \begin{array}{llc} \infty & : & \|\theta\|_1 > 1 \\ 0 & : & \|\theta\|_1 \leq 1 \end{array} \right.\]
%% If we use Method 1, we then end up with 
%% \[ u_{t}(x) \propto \exp\left\{-\frac{1}{\alpha}\overline{\theta}_{t}^T\phi(x)\right\}, \]
%% where $\overline{\theta}_{t} = \frac{1}{t} \sum_{s \le t} \theta_{s}$. Hence, $h(u) + \bE_{\mu}[\theta^T\phi(x)] - R(\theta)$ is no longer maximized at the boundary (when $\tau > 0$). We can instead consider the following sequence of distributions:
%% \begin{enumerate}
%% \item $\theta_t \in \argmax_{\|\theta\|_1 \leq 1} \theta^T (\bE_{u_t}[\phi(x)]-\bar{\phi})$
%% \item $u_{t+1}(x) = Z(\overline{\theta}_{t}, \tau)^{-1} \exp\left\{-\frac{1}{\tau}\overline{\theta}_{t}^T\phi(x)\right\}$
%% \end{enumerate}
%% Here $Z(\overline{\theta}_{t}, \tau) = \int\exp\left\{-\frac{1}{\tau}\overline{\theta}_{t}^T\phi(x)\right\} \dee x$. 
%% 
%% The entropic regularizer is not strongly convex but a more general analysis involving 
%% Bregman divergence can nevertheless yield a convergence bound in this case. We thus 
%% get something that looks like herding but actually gives us an exponential family 
%% type distribution. In particular, we have a Gibbs distribution with temperature $\tau$, and 
%% in the limit as the temperature goes to zero, we recover herding, since 
%% \[ \lim_{\tau \to 0} u_{t+1}(x) = \delta_{x^{*}_{t+1}}, \]
%% where $x^{*}_{t+1} = \argmin_{x} \overline{\theta}_{t}^T\phi(x)$. 
%% This fact is reminiscent of Welling's original derivation of herding, in which 
%% he took the zero temperature limit of the gradient descent algorithm for the optimization problem 
%% \[ \argmin_{u} h(u) \quad \text{s.t.}~\bE_{u}[\phi(x)] - \bar\phi = 0. \]
%% This optimization is quite similar to the entropic optimization considered here, which 
%% we can write as 
%% \begin{align*}
%% \argmin_{u}& \argmax_{|\theta\|_1 \leq 1} h(u) + \theta^{T}(\bE_{u}[\phi(x)] - \bar \phi)  \\
%% &= \argmin_{u} h(u) + \|\bE_{u}[\phi(x)] - \bar\phi\|_{2}. 
%% \end{align*}
%% In the entropic formulation, the expectation constraint $\bE_{u}[\phi(x)] - \bar\phi = 0$ becomes
%% soft, though the two optimizations are equivalent in the zero-temperature limit. 
%% 
%% We can view the entropic regularization
%% with non-zero temperature as an alternative approximation to herding. If at each step we
%% retain a sample $x_{t} \sim u_{t}$, then the sequence $(x_{t})_{t}$ converges to the 
%% herding sample path as $\tau \to 0$. So $(x_{t})_{t}$ with non-zero temperature approximate
%% herding samples. \NA{JHH: I'm not sure how useful this all is, other than to make contact with 
%% the original paper, as I think the Gibbs distribution was exactly what Max was trying to 
%% avoid by taking the zero-temperature limit.}

\section{Conclusion}
\label{sec:conclusion}

%\bibliography{example_paper}
%\bibliographystyle{icml2013}

\end{document} 


% This document was modified from the file originally made available by
% Pat Langley and Andrea Danyluk for ICML-2K. This version was
% created by Lise Getoor and Tobias Scheffer, it was slightly modified  
% from the 2010 version by Thorsten Joachims & Johannes Fuernkranz, 
% slightly modified from the 2009 version by Kiri Wagstaff and 
% Sam Roweis's 2008 version, which is slightly modified from 
% Prasad Tadepalli's 2007 version which is a lightly 
% changed version of the previous year's version by Andrew Moore, 
% which was in turn edited from those of Kristian Kersting and 
% Codrina Lauth. Alex Smola contributed to the algorithmic style files.  
