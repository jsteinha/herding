\documentclass[paper.tex]{subfiles}

\begin{document}
\section{An Upper Bound in Infinite Dimensions}
\label{sec:chen-proofs}

An intriguing aspect of classical herding is that it is known to have a convergence rate of 
$O(1/T^2)$ under suitable conditions, as was shown in \citet{Chen:2010b}. In this section, 
we will need to assume that $h(x)$ is the indicator function for $U$ (i.e. $0$ on $U$ and 
$\infty$ outside of it) and that $R(\theta) = \ip{u_0}{\theta} + R_0(\theta)$, where 
$R_0$ exhibits \emph{quadratic scaling}: $R_0(s\theta) = s^2R_0(\theta)$. Note that 
this implies that $R_0^*(u-u_0) = R^*(u)$.
For simplicity 
we will focus on the case $\alpha_t = 1$. Then we have 
the following bound:
\begin{theorem}
\label{thm:chen}
Suppose that $R$ is strongly convex with respect to the norm $\|\cdot\|$ and 
let $r_* \eqdef \sup_{u \in U} \|u\|_*$. Also suppose that $h \equiv 0$ and 
that $R_0$ exhibits quadratic scaling as above. Define 
\[ \gamma^* \eqdef \inf_{\theta \in \Theta} \sup_{u \in U} \frac{\ip{-\theta}{u-u_0}}{\sqrt{R(\theta)}}. \]
Then $R(\theta_{t+1}) \leq \frac{2r_*^2+(2r_*^2/\gamma_*)^2}{t^2}$ as long 
as $\gamma^* > 0$.
\end{theorem}
This corresponds to Proposition 1 of \citet{Chen:2010b} for $R(\theta) = \frac{1}{2}\|\theta\|_2^2$. Note that if there is an $l^2$ ball of radius $\epsilon$ 
around $u_0$ that is contained in $u$ then $\gamma^* \geq \frac{\epsilon}{r_*}$ 
in this case, which is how Chen's result follows from ours.
\begin{proof}[Proof of Theorem~\ref{thm:chen}]
The key quantity we will analyze is $(t-1)^2R(\theta_t)$. Note that we have
\begin{align*}
\lefteqn{(t-1)^2R(\theta_t) - t^2R(\theta_{t+1})} \\
 &= (t-1)^2R^*(\hat{u}_{t-1})-t^2R^*(\hat{u}_t) \\
 &= (t-1)^2R_0^*(\hat{u}_{t-1})-t^2R_0^*(\hat{u}_t-z) \\
 &= (t-1)^2\left[R^*(\hat{u}_{t-1})-R_0^*((t/(t-1))(\hat{u}_t-z))\right] \\
 &= (t-1)^2\left[R^*(\hat{u}_{t-1})-R^*\left(\frac{t\hat{u}_t-z}{t-1}\right)\right] \\
 &= (t-1)^2\left[\ip{\frac{\partial R^*}{\partial \hat{u}_{t-1}}}{\hat{u}_{t-1}-\frac{t\hat{u}_t-z}{t-1}}-D_{R^*}\left(\frac{t\hat{u}_t-z}{t-1} \middle\| \hat{u}_{t-1}\right)\right] \\
 &\geq (t-1)^2\left[\ip{\theta_t}{-\frac{1}{t-1}(u_t-z)}-\frac{1}{2}\left\|\frac{u_t-z}{t-1}\right\|_*^2\right] \\
 &= -(t-1)\ip{\theta_t}{u_t-z}-\frac{1}{2}\|u_t-z\|_*^2 \\
 &= -(t-1)\left[\inf_{u \in U} \ip{\theta_t}{u} - \ip{\theta_t}{z}\right]-\frac{1}{2}\|u_t-z\|_*^2 \\
 &\geq (t-1)\sqrt{R(\theta_t)}\left[\inf_{\theta} \sup_{u \in U} \ip{-\theta}{u-u_0}\right]-2r_*^2 \\
 &= \sqrt{(t-1)^2R(\theta_t)}\gamma^*-2r_*^2.
\end{align*}
There are now two possibilities for each $t$:
\begin{enumerate}
\item $t^2R(\theta_{t+1}) < (t-1)^2R(\theta_t)$, or
\item $(t-1)^2R(\theta_t) \leq \frac{4r_*^4}{(\gamma^*)^2}$.
\end{enumerate}
In particular, consider the value of $t$ for which 
$t^2R(\theta_{t+1})$ is maximized. Then by assumption 
case $1$ cannot hold and so $(t-1)^2R(\theta_t) \leq \frac{4r_*^4}{(\gamma^*)^2}$.
But then by applying the above sequence of inequalities 
we also have 
\begin{align*}
\lefteqn{t^2R(\theta_{t+1}) - (t-1)^2R(\theta_t)} \\
 &\leq 2r_*^2 - \sqrt{(t-1)^2R(\theta_t)}\gamma^* \\
 &\leq 2r_*^2,
\end{align*}
hence $t^2R(\theta_{t+1}) \leq 2r_*^2 + \frac{4r_*^4}{(\gamma_*)^2}$. 
Since $t^2R(\theta_{t+1})$ was maximal by assumption, we have
$R(\theta_{t+1}) \leq \frac{2r_*^2+(2r_*/\gamma_*)^2}{t^2}$, which 
completes the proof.
\end{proof}

\end{document}
