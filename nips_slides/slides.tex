\documentclass{beamer}
\usetheme{Madrid} % My favorite!
%\usetheme{Boadilla} % Pretty neat, soft color.
%\usetheme{default}
%\usetheme{Warsaw}
%\usetheme{Bergen} % This template has nagivation on the left
%\usetheme{Frankfurt} % Similar to the default 
%with an extra region at the top.
%\usecolortheme{seahorse} % Simple and clean template
%\usetheme{Darmstadt} % not so good
% Uncomment the following line if you want %
% page numbers and using Warsaw theme%
% \setbeamertemplate{footline}[page number]
%\setbeamercovered{transparent}
\setbeamercovered{invisible}
% To remove the navigation symbols from 
% the bottom of slides%
\setbeamertemplate{navigation symbols}{} 
%
\usepackage{graphicx}
%\usepackage{bm}         % For typesetting bold math (not \mathbold)
%\logo{\includegraphics[height=0.6cm]{yourlogo.eps}}
%
\usepackage{import, subfiles}
\usepackage[absolute,overlay]{textpos}
\usepackage{mathtools}
\usepackage{hyperref,url}
\usepackage{amsmath,amssymb,amsthm}
\usepackage{tikz}
\usepackage{float,graphicx}
\usepackage{stmaryrd,wasysym,clrscode}
\usepackage{etex,etoolbox}
\usepackage{ifthen}
\usetikzlibrary{patterns,positioning}

\DeclareMathOperator{\Tr}{Tr}
\def\argmax{\operatornamewithlimits{arg\,max}}
\def\argmin{\operatornamewithlimits{arg\,min}}
\newcommand{\Fit}{\mathtt{Fit}}
\newcommand{\DP}{D}
\newcommand{\Subtree}{\operatorname{Subtree}}
\newcommand{\E}{\mathcal{E}}
\newcommand{\eqdef}{\stackrel{\mathrm{def}}{=}}
\newcommand{\bI}{\mathbb{I}}
\newcommand{\bE}{\mathbb{E}}
\newcommand{\bP}{\mathbb{P}}
\newcommand{\bR}{\mathbb{R}}
\newcommand{\sF}{\mathcal{F}}
\newcommand{\sC}{\mathcal{C}}
\newcommand{\C}{\mathcal{C}}
\newcommand{\sB}{\mathcal{B}}
\newcommand{\sI}{\mathcal{I}}
\newcommand{\sP}{\mathcal{P}}
\newcommand{\sX}{\mathcal{X}}
\newcommand{\meet}{\wedge}
\newcommand{\RE}[2]{\operatorname{RE}\left(#1 \ \| \ #2\right)}
\newcommand{\K}{\operatorname{KL}}
\newcommand{\KL}[2]{\operatorname{KL}\left(#1 \ \middle\| \ #2\right)}
\newcommand{\KLs}[3]{\operatorname{KL}_{#1}\left(#2 \ \middle\| \ #3\right)}
\newcommand{\KLm}[2]{\operatorname{KL}_m\left(#1 \ \| \ #2\right)}
\newcommand{\score}[2]{\operatorname{score}\left(#1 \ \| \ #2\right)}
\newcommand{\phih}{\hat{\phi}}
\newcommand{\psih}{\hat{\psi}}
\newcommand{\pw}{\pi}
\newcommand{\B}{B}
\newcommand{\F}{\mathbb{S}}
\newcommand{\bmu}{\mathbb{U}}
\DeclareMathOperator{\supp}{supp}
\DeclareMathOperator{\loc}{loc}
\DeclareMathOperator{\lub}{lub}
\newcommand{\mloc}{m_{\loc}}
\newcommand{\atom}[1]{#1^{\circ}}
\newcommand{\atoms}[1]{#1^{\circ}}
\newcommand{\stitch}[2]{\overline{#1}^{#2}}
\newcommand{\relaxx}[2]{\widehat{#1}^{#2}}
\DeclareMathOperator{\sn}{sn}
\DeclareMathOperator{\Uniform}{Uniform}




\title[Greedy First-Order Optimization]{A Greedy Framework for First-Order Optimization}
\author[JHH and JS]{Jonathan Huggins \inst{1} \and Jacob Steinhardt \inst{2}}
\institute[MIT, Stanford]{\inst{1} Massachusetts Institute of Technology \and \inst{2} Stanford University}
\date{Dec 10, 2013}
% \today will show current date. 
% Alternatively, you can specify a date.
%
\begin{document}
%
\begin{frame}
\titlepage
\end{frame}
%
\newcommand{\reminder}{
\begin{textblock}{14}(9.5,1.9)
\fbox{$L(u,\theta) \eqdef h(u) + u^T\theta - R(\theta)$}
\end{textblock}}
\begin{frame}
\frametitle{Motivation}
We want to solve the following \emph{saddle point} problem:
\[ \min_{u} \max_{\theta} L(u,\theta), \]
where $L(u,\theta) = h(u) + u^T\theta - R(\theta)$. (Assume $h, R$ convex.)
\pause
\vskip 0.2in
Tie-in with optimization: can think of as minimizing
\[ L(u) \eqdef \max_{\theta} L(u,\theta) = h(u) + R^*(u). \]
%Think of as two-player game between $u$ and $\theta$ with payoff 
%$L(u,\theta) \eqdef h(u) + u^T\theta - R(\theta)$.
\only<3>{\reminder}
\end{frame}
%
\begin{frame}
\frametitle{Being (Too) Greedy}
\reminder
Let's try the following updates:
\begin{align*}
u_t\,\,\,\,\,\, &= \argmin_{u} L(u,\theta_t) \\
\theta_{t+1}   &= \argmax_{\theta} L(u_t, \theta)
\end{align*}
``iterative best response''
\vskip 0.2in
\pause
Issue: let $u,\theta \in \bR$, $h(u) = \frac{1}{2}u^2$, $R(\theta) = \frac{1}{2}\theta^2$. 
Then:
\begin{align*}
u_t\,\,\,\,\,\, &= \argmin_u \left[\frac{1}{2}u^2 + u\theta\right]             = -\theta_t \\
\theta_{t+1}    &= \argmax_{\theta} \left[u\theta - \frac{1}{2}\theta^2\right] = u_t
\end{align*}
\textcolor{red}{OSCILLATION}
\end{frame}

\begin{frame}
\frametitle{Being Just Greedy Enough}
\reminder
Can get what we want if we replace $u_t$ with $\hat{u}_t \eqdef \frac{1}{t} \sum_{s=1}^t u_s$:
\begin{align*}
u_t\,\,\,\,\,\, &= \argmin_{u} L(u,\theta_t) \\
\theta_{t+1}    &= \argmax_{\theta} L(\hat{u}_t, \theta)
\end{align*}
\pause
\begin{theorem}
If $R$ is strongly convex then $|L(u_t,\theta_t)-L(u^*,\theta^*)| \leq O\left(\frac{\log(T)}{T}\right)$.
\end{theorem}
Note: can get $O(1/T)$ convergence if we use a weighted average for $\hat{u}_t$.
\end{frame}

\begin{frame}
\frametitle{Frank-Wolfe}
\reminder
We get Frank-Wolfe for $h \equiv 0$:
\begin{align*}
u_t\,\,\,\,\,\, &= \argmin_u L(u, \theta_t) = \Aboxed{\argmin_u u^T\theta_t} \\
\theta_{t+1}    &= \argmax_{\theta} L(\hat{u}_t, \theta) = \argmax_{\theta} \hat{u}_t^T\theta - R(\theta) = \Aboxed{\partial R^*(\hat{u}_t)}
\end{align*}
\pause
General updates:
\begin{align*}
u_t\,\,\,\,\,\, &= \partial h^*(-\theta_t) \\
\theta_{t+1} &= \partial R^*(\hat{u}_t).
\end{align*}
\end{frame}

\begin{frame}
\frametitle{Applications}
\begin{tabular}{|c|c|}
\hline
$L(u,\theta)$ & Algorithm \\ \hline
$h(u) + u^T\theta - f^*(\theta)$ & mirror descent \\ \hline
$\|u\|_1 + (Au-y)^T\theta - \frac{1}{2}\|\theta\|_2^2$ & thresholded Frank-Wolfe \\ \hline
$\Tr(A^TX) + \sum_{i=1}^n y_i(X_{ii} -1 -\eta \log y_i)$ & AHK low-rank SDP \\ \hline
$\bE_{x \sim \mu}[\theta^T(\phi(x)-\bar{\phi})] - \frac{1}{q}\|\theta\|_q^q$ & $q$-herding \\ \hline
\end{tabular}
(Note: some of the entries above use a dual version of the algorithm.)
\end{frame}
\end{document} 
