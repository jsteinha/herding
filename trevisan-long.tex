\documentclass[11pt]{article}
\usepackage{import}
\usepackage{hyperref,url}
\usepackage{amsmath,amssymb,amsthm}
\usepackage{tikz}
\usepackage{float,caption,subcaption,graphicx}
\usepackage{stmaryrd,wasysym,clrscode}
\usepackage{etex,etoolbox}
\usepackage{ifthen}
\usetikzlibrary{patterns,positioning}

\newcommand{\lc}[1]{#1_{\mathrm{loc}}}
\newcommand{\eq}[1]{\stackrel{\mathrm{#1}}{=}}
\DeclareMathOperator{\Var}{Var}
\DeclareMathOperator{\MMD}{MMD}
\newcommand{\MMDr}{\tilde{\MMD}}
\DeclareMathOperator{\Tr}{Tr}
\newcommand{\inner}[2]{\langle #1, #2 \rangle}
\newcommand{\E}{\mathcal{E}}
\newcommand{\eqdef}{\stackrel{\mathrm{def}}{=}}
\newcommand{\bP}{\mathbb{P}}
\newcommand{\bI}{\mathbb{I}}
\newcommand{\bE}{\mathbb{E}}
\newcommand{\sF}{\mathcal{F}}
\newcommand{\sD}{\mathcal{D}}
\newcommand{\sC}{\mathcal{C}}
\newcommand{\C}{\mathcal{C}}
\newcommand{\sB}{\mathcal{B}}
\newcommand{\bR}{\mathbb{R}}
\newcommand{\sA}{\mathcal{A}}
\newcommand{\sI}{\mathcal{I}}
\newcommand{\sP}{\mathcal{P}}
\newcommand{\sX}{\mathcal{X}}
\newcommand{\sY}{\mathcal{Y}}
\newcommand{\sS}{\mathcal{S}}
\newcommand{\sJ}{\mathcal{J}}
\newcommand{\sR}{\mathcal{R}}
\newcommand{\sU}{\mathcal{U}}
\newcommand{\sN}{\mathcal{N}}
\newcommand{\sK}{\mathcal{K}}
\newcommand{\sM}{\mathcal{M}}
\newcommand{\meet}{\wedge}
\newcommand{\RE}[2]{\operatorname{RE}\left(#1 \ \| \ #2\right)}
\newcommand{\KL}[2]{\operatorname{KL}\left(#1 \ \| \ #2\right)}
\newcommand{\KLm}[2]{\operatorname{KL}_m\left(#1 \ \| \ #2\right)}
\newcommand{\score}[2]{\operatorname{score}\left(#1 \ \| \ #2\right)}
\newcommand{\phih}{\hat{\phi}}
\newcommand{\psih}{\hat{\psi}}
\DeclareMathOperator{\supp}{supp}
\DeclareMathOperator{\loc}{loc}
\DeclareMathOperator{\lub}{lub}
\newcommand{\atom}[1]{#1^{\circ}}
\newcommand{\stitch}[2]{\overline{#1}^{#2}}
\def\argmax{\operatornamewithlimits{arg\,max}}
\def\argmin{\operatornamewithlimits{arg\,min}}

\newtheorem{theorem}{Theorem}[section]
\newtheorem{lemma}[theorem]{Lemma}
\newtheorem{proposition}[theorem]{Proposition}
\newtheorem{corollary}[theorem]{Corollary}
\newtheorem{assumption}[theorem]{Assumption}
\theoremstyle{definition}
\newtheorem{example}[theorem]{Example}
\newtheorem{definition}[theorem]{Definition}
\newtheorem{remark}[theorem]{Remark}
\newtheorem{property}[theorem]{Property}

\def\ci{\perp\!\!\!\perp}

\usepackage{fullpage}

\title{Computational Indistinguishability via Boosting}
\author{Jacob Steinhardt\footnote{Based on Trevisan et al, 2010.}}
\begin{document}
\maketitle

\section{Setup}
Suppose we have a family $\sF$ of functions $f : X \to [0,1]$. We 
say that $h$ has complexity $C$ relative to $f$ if it can be obtained 
from elements of $f$ via at most $C$ of the following operations:
\begin{enumerate}
\item addition: $f,g \mapsto f+g$
\item multiplication: $f,g \mapsto f\cdot g$
\item multiplication by scalars: $c,f \mapsto c \cdot f$
\item thresholding: $f \mapsto \bI[f(x) \leq t]$
\end{enumerate}
Note that two-sided thresholding such as $\bI[s \leq f(x) \leq t]$ can be 
expressed as $\bI[f(x) \leq t] \cdot \bI[-f(x) \leq -s]$ and so has complexity 
$4$ relative to $\sF$. We will denote by $\sF[C]$ the space of $h : X \to [0,1]$ 
with complexity at most $C$ relative to $\sF$.

Given a distribution $\mu$ on $X$, we say that $g$ and $h$ are 
\emph{$\epsilon$-indistinguishable} relative to $\sF$ if 
$|\bE_{\mu}[f(x)(g(x)-h(x))] \leq \epsilon$ for all $f \in \sF$. 

\section{Main Result}
The main result is the following:

\begin{theorem}
\label{thm:luca}
Let $\sF$ be any family of functions $f : X \to [0,1]$, let $\mu$ be a 
distribution on $X$, and let $g$ be any function $g : X \to [0,1]$. Then 
there exists an $h$ with complexity $O(1/\epsilon^2)$ relative to $\sF$ that is 
$\epsilon$-indistinguishable from $g$ relative to $\sF$. Furthermore, $\bE_{\mu}[g(x)] = \bE_{\mu}[h(x)]$.
\end{theorem}

\begin{proof}
Assume WLOG that $\sF = -\sF$, so that we need only prove indistinguishability 
in one direction. Also, assume that $1 \in \sF$, so that $\bE_{\mu}[g(x)] \approx \bE_{\mu}[h(x)]$ (we can 
make them exactly equal by slightly adjusting $h$ post-hoc).

We use the following stochastic gradient descent algorithm:
\begin{itemize}
\item $h_0 = 0$
\item $f_i \in \argmax_{f \in \sF} \bE_{\mu}[f(x)(h_i(x)-g(x))]$
\item $h_{i+1} = \min(\eta \sum_{j \leq i} f_j, 1)$
\end{itemize}
This is online gradient descent with regularizer $\frac{1}{2\eta}\bE_{\mu}[h(x)^2]$, so we get the regret bound:
\begin{align*}
\sum_{i=0}^{k-1} \bE_{\mu}[f_i(x)(h_i(x)-g(x))] &\leq \sum_{i=0}^k \bE_{\mu}[f_i(x)(g(x)-g(x))] + \frac{1}{2\eta}\bE_{\mu}[g(x)^2] + \frac{\eta}{2}\sum_{i=0}^{k-1} \bE_{\mu}[f_i(x)^2] \\
 &\leq \frac{1}{2\eta} + \frac{k\eta}{2}.
\end{align*}
Setting $\eta = \frac{1}{\sqrt{k}}$ yields 
\[ \sum_{i=0}^k \bE_{\mu}[(h_i(x)-g(x))f_i(x)] \leq \sqrt{k}, \]
meaning that $\bE_{\mu}[(h_i(x)-g(x))f_i(x)] \leq \frac{1}{\sqrt{k}}$ for some $i < k$. 
Since $f_i$ is by construction the maximizer of $\bE[(h_i(x)-g(x))f(x)]$, we have that 
$\bE[(h_i(x)-g(x))f(x)] \leq \frac{1}{\sqrt{k}}$ for all $f \in \sF$ for some $i < k$. Letting 
$k = \frac{1}{\epsilon^2}$ then gives us $\epsilon$-distinguishability relative to $\sF$.

Since $\min(\cdot, 1)$ can be implemented in $O(1)$ steps and $k = \frac{1}{\epsilon^2}$, we 
then have that $h_i$ has complexity $O\left(\frac{1}{\epsilon^2}\right)$ relative to $\sF$, which 
proves the theorem.
\end{proof}

\begin{remark}
Note that the theorem still holds (possibly with different constants) if we replace $[0,1]$ by any 
bounded set; in particular, with $[-1,1]$ (this fact is used in Trevisan et al. although here we will 
only use $[0,1]$).
\end{remark}

%We can also prove a slightly strengthened version of the theorem:
%\begin{theorem}
%\label{thm:luca-strong}
%Under the same conditions as Theorem~\ref{thm:luca}, we can obtain a 
%function $h : X \to \mathbb{R}_{\geq 0}$ that is a non-negative \emph{linear} 
%combination of $O(1/\epsilon^2)$ members of $\sF$, and such that $g$, $h$, 
%and $\min(h,1)$ are all mutually $\epsilon$-indistinguishable. Furthermore, 
%$\bE_{\mu}[g(x)] = \bE_{\mu}[\min(h(x),1)]$.
%\end{theorem}
%\begin{remark}
%In this version of the theorem, the proof will no longer hold if we replace $[0,1]$ by $[-1,1]$: 
%the non-negativity of the co-domain is important.
%\end{remark}
%\begin{proof}
%Perform stochastic gradient descent as before, but now with the loss function 
%$l_f(h) = \bE_{\mu}\left[f(x) \cdot \frac{h(x) + \min(h(x),1)}{2}\right]$. Also, for this 
%version, do not project $h$ but simply perform the gradient update $h_{i+1} = h_i + \eta \partial_h l_f(h)$. 
%If we actually compute this update, we get
%\[ h_{i+1}(x) = f(x) \cdot \frac{1 + \bI[h_i(x) < 1]}{2}. \]
%Note that $h$ and $f$ are both non-negative, so we can conclude the following:
%\begin{itemize}
%\item If $l_f(h) < \frac{\epsilon}{2}$, then $\bE_{\mu}[f(x)h(x)] < \epsilon$ and 
%      $\bE_{\mu}[f(x)\min(h(x),1)] < \epsilon$. Therefore, 
%\end{itemize}
%\end{proof}

\section{Corollaries}
\subsection{Dense Model Theorem}
Theorem~\ref{thm:luca} has many interesting corollaries. The first is the dense model theorem. Call 
a distribution $p$ \emph{$\delta$-dense} relative to $q$ if $p(x) \leq \frac{q(x)}{\delta}$: that is, 
$p$ is $\delta$-dense if it is never too much more likely than $q$. This implies in particular that 
there is a rejection sampler for $p$ with proposal distribution $q$ and acceptance rate at least $\delta$.

\begin{theorem}[Dense Model Theorem]
Suppose that $X$ is a finite set, let $p_U$ be the uniform distribution 
on $X$, and let $\sF$ be a family of functions $f : X \to [0,1]$. 
Let $p_D$ and $p_R$ be distributions such that $p_R$ and $p_U$ have moments in $\sF[C]$ differing 
by at most $\epsilon$: $|\bE_{p_R}[f(x)]-\bE_{p_U}[f(x)]| \leq \epsilon$ for all $f \in \sF[C]$.
Suppose also that $p_D$ is $\delta$-dense in $p_R$. Then, if $C \geq O(1/\epsilon^2)$, 
there is a distribution $p_M$ that is $(\delta-\epsilon)$-dense in $p_U$, and whose moments in $\sF$ 
differ from $p_D$ by at most $\frac{3\epsilon}{\delta}$.
\end{theorem}

\begin{proof}
Let $g(x) = 1-\delta\frac{p_D(x)}{p_R(x)}$. Let $h(x)$ be an $\epsilon$-indistinguishable function 
with respect to the measure $p_R$ and the family $\sF$, and define $p_M(x) \propto 1-h(x)$.

First, let us compute the normalization constant of $p_M$. We have:
\begin{align*}
\sum_{x} p_M(x) &= |X|\bE_{p_U}[(1-h(x))] \\
 &= |X|(\bE_{p_R}[(1-h(x))] \pm \epsilon) \\
 &= |X|(\bE_{p_R}[(1-g(x))] \pm \epsilon) \\
 &= |X|(\delta \pm \epsilon).
\end{align*}

It follows that $p_M(x) \leq \frac{1-h(x)}{|X|(\delta-\epsilon)} \leq \frac{1}{|X|(\delta-\epsilon)}$ and 
hence $p_M$ is $(\delta-\epsilon)$-dense in $p_U$.

To get indistinguishability, we need to show that $|\bE_{p_D}[f(x)]-\bE_{p_M}[f(x)]|$ is always small. 
We do this by showing that $p_D(x)$ and $p_M(x)$ are both close to $\frac{1-h(x)}{\delta}$. Indeed, we have
\begin{align*}
\left|\bE_{p_D}[f(x)]-\bE_{p_U}\left[\frac{1-h(x)}{\delta}f(x)\right]\right| &\leq \left|\bE_{p_D}[f(x)]-\bE_{p_R}\left[\frac{1-h(x)}{\delta}f(x)\right]\right| + \frac{\epsilon}{\delta} \\
 &\leq \left|\bE_{p_D}[f(x)]-\bE_{p_R}\left[\frac{1-g(x)}{\delta}f(x)\right]\right| + \frac{2\epsilon}{\delta} \\
 &= \left|\bE_{p_D}[f(x)]-\bE_{p_R}\left[\frac{p_D(x)}{p_R(x)}f(x)\right]\right| + \frac{2\epsilon}{\delta} \\
 &= \frac{2\epsilon}{\delta},
\end{align*}
as well as
\begin{align*}
\left|\bE_{p_M}[f(x)] - \bE_{p_U}\left[\frac{1-h(x)}{\delta}f(x)\right]\right| &= \left|\bE_{p_M}[f(x)] - \frac{\sum_{x} 1-h(x)}{\delta|X|} \bE_{p_M}[f(x)]\right| \\
 &= \left|\bE_{p_M}[f(x)] - \frac{(\delta \pm \epsilon)|X|}{\delta |X|} \bE_{p_M}[f(x)] \right| \\
 &\leq \frac{\epsilon}{\delta} |\bE_{p_M}[f(x)]| \\
 &\leq \frac{\epsilon}{\delta}.
\end{align*}
These two inequalities complete the proof.
\end{proof}

\begin{remark}
Note that the construction also shows that $p_M \in \sF[C]$. Since $p_M$ has high density relative to $p_U$, 
we can build a rejection sampler for $p_M$ using the uniform distribution as a proposal, and hence sample $p_M$ 
efficiently. If we set $R = U$, then this implies in particular that every $\delta$-dense distribution has 
an indistinguishable distribution that can be efficiently sampled.
\end{remark}

%\begin{remark}
%Note that we only need to multiply functions in $\sF$ once, when we consider $\frac{1-h(x)}{\delta} f(x)$. 
%So suppose that $\sF$ is defined by some collection $\phi : X \to \{0,1\}^d$ of binary features. Also 
%imagine that $R$ and $U$ have identical \emph{second} moments with respect to $\phi$: 
%$\bE_{p_U}[\phi(x)\phi(x)^T] = \bE_{p_R}[\phi(x)\phi(x)^T]$. (We'll also assume that one of the features 
%is identically equal to $1$, so that this also implies that $R$ and $U$ have identical first moments.)
%
%By the construction in Theorem~\ref{thm:luca}, $h(x) = \min(\eta \sum_{i=1}^k \phi_i(x), 1)$. 
%For the result of the dense model theorem to hold, we need $\bE_{p_U}[h(x)\phi(x)] = \bE_{p_R}[h(x)\phi(x)]$, 
%or equivalently
%\[ \bE_{p_U}[\min(\eta \sum_{i=1}^k \phi_i(x), 1)\phi(x)] = \bE_{p_R}[\min(\eta \sum_{i=1}^k \phi_i(x), 1)\phi(x)]. \]
%Without the $\min$, we would be in good shape. It turns out we can modify our construction so that the $\min$ affects 
%things by at most $\epsilon$, so that these expectations are indeed within $\epsilon$ of each other.
%\end{remark}
\subsection{Szemeredi Regularity Lemma}
\begin{theorem}[Szemeredi]
Let $G$ be a graph with edge set $E$. Then there exists a partition of $V$ into $M$ components $V_1,\ldots,V_M$ and a 
function $w : \{1,\ldots,M\} \times \{1,\ldots,M\} \to [0,1]$ such that if $A \subset V_i$ and $B \subset V_j$ then 
the number of edges $e(A,B)$ between $A$ and $B$ satisfies $|A||B|w(i,j)-\epsilon|V|^2 \leq e(A,B) \leq |A||B|w(i,j) + \epsilon|V|^2$. Furthermore, 
we can take $M = 2^{O(1/\epsilon^2)}$.
\end{theorem}

\begin{proof}
Let $\sF$ consist of all functions from $V \times V$ to $\{0,1\}$ of the form $f_{A,B}(u,v) = \bI[u \in A] \bI[v \in B]$. Let $g(u,v) = \bI[(u,v) \in E]$. 
Then there exists some $h = \sum_{i=1}^k c_i f_{A_i,B_i}$ with $k = O(1/\epsilon^2)$ that is $\epsilon$-indistinguishable from $g$ with respect to $\sF$. 
We can then partition $V$ into $2^{2k}$ sets $V_1,\ldots,V_{2^{2k}}$ such that $h$ is piecewise constant on each $V_i \times V_j$ (just let the $V_j$ consist of all 
intersections of the $A_i$ and the $B_i$ or their complements). We can then define $w(i,j)$ to be the constant value that $h$ takes on $V_i \times V_j$.
In particular, indistinguishability gives us $|\sum_{u \in A, v \in B} h(u,v) - g(u,v)| \leq \epsilon |V|^2$, 
or equivalently $|w(i,j)|A||B|-e(A,B)| \leq \epsilon |V|^2$, which proves the theorem.
\end{proof}

\subsection{Impaggliazo Hard-Core Set Lemma}
\begin{theorem}
Let $X$ be a set and $p_U$ the uniform distribution over that 
set. Let $\sF$ be a family of boolean functions and let 
$g$ be a boolean function such that $p_U[f(x)=g(x)] \leq 1-\delta$ 
for all $f \in \sF[C]$, with $C \geq O(1/(\delta\epsilon)^2)$. Then 
there exists a distribution $p_M \in \sF[C]$ such that $p_M(x) \leq \frac{1}{\delta |X|}$ 
for all $x \in X$, and 
$p_M[f(x)=g(x)] \leq \frac{1}{2}+\epsilon$ for all $f \in \sF$.
\end{theorem}
Intuitively, this means that if $g$ is somewhat hard to match by any function 
with complexity $C$ relative to $\sF$, then there is some ``core'' of $X$ that 
is hard for any element of $\sF$ to match at all.
\begin{proof}
Let $h \in \sF[C]$ be $(\delta\epsilon)$-indistinguishable from $g$ 
relative to $\sF$. Let $p_M(x) \propto |g(x)-h(x)|$. First, we note that 
$\bE_{p_U}[|g(x)-h(x)|] \geq \delta$, or else $\bI[h(x) \leq t]$ would match 
$g(x)$ on more than $1-\delta$ of the inputs for some $t$. Therefore, 
$\sum_{x \in X} |g(x)-h(x)| \geq \delta |X|$ and hence $p_M(x) \leq \frac{|g(x)-h(x)|}{\delta |X|} \leq \frac{1}{\delta |X|}$, 
which establishes the first condition.

To finish, note the identity 
\begin{align*}
|g(x)-h(x)| \bI[f(x)=g(x)] &= \frac{1}{2}|g(x)-h(x)| + \frac{1}{2}|g(x)-h(x)|(-1)^{f(x)-g(x)} \\
 &= \frac{1}{2}|g(x)-h(x)| - \frac{1}{2}(g(x)-h(x))(-1)^{f(x)-g(x)+g(x)} \\ 
 &= \frac{1}{2}|g(x)-h(x)| - \frac{1}{2}(g(x)-h(x))(-1)^{f(x)} \\
 &= \frac{1}{2}|g(x)-h(x)| + \left(f(x)-\frac{1}{2}\right) (g(x)-h(x)).
\end{align*}
It then follows that 
\begin{align*}
p_M[\bI[f(x)=g(x)]] &= \frac{\sum_{x} \frac{1}{2}|g(x)-h(x)| + \sum_{x} \left(f(x)-\frac{1}{2}\right) (g(x)-h(x))}{\sum_{x} |g(x)-h(x)|} \\
 &= \frac{1}{2} + \frac{\sum_{x} f(x)(g(x)-h(x))}{\sum_{x} |g(x)-h(x)|} \\
 &\leq \frac{1}{2} \frac{\epsilon\delta}{\delta} \\
 &= \frac{1}{2} + \epsilon,
\end{align*}
which yields the desired result.
\end{proof}
\end{document}
